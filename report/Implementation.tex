
% Default to the notebook output style

    


% Inherit from the specified cell style.




    
\documentclass[11pt]{article}

    
    
    \usepackage[T1]{fontenc}
    % Nicer default font (+ math font) than Computer Modern for most use cases
    \usepackage{mathpazo}
    \pagenumbering{gobble}

    % Basic figure setup, for now with no caption control since it's done
    % automatically by Pandoc (which extracts ![](path) syntax from Markdown).
    \usepackage{graphicx}
    % We will generate all images so they have a width \maxwidth. This means
    % that they will get their normal width if they fit onto the page, but
    % are scaled down if they would overflow the margins.
    \makeatletter
    \def\maxwidth{\ifdim\Gin@nat@width>\linewidth\linewidth
    \else\Gin@nat@width\fi}
    \makeatother
    \let\Oldincludegraphics\includegraphics
    % Set max figure width to be 80% of text width, for now hardcoded.
    \renewcommand{\includegraphics}[1]{\Oldincludegraphics[width=.8\maxwidth]{#1}}
    % Ensure that by default, figures have no caption (until we provide a
    % proper Figure object with a Caption API and a way to capture that
    % in the conversion process - todo).
    \usepackage{caption}
    \DeclareCaptionLabelFormat{nolabel}{}
    \captionsetup{labelformat=nolabel}

    \usepackage{adjustbox} % Used to constrain images to a maximum size 
    \usepackage{xcolor} % Allow colors to be defined
    \usepackage{enumerate} % Needed for markdown enumerations to work
    \usepackage{geometry} % Used to adjust the document margins
    \usepackage{amsmath} % Equations
    \usepackage{amssymb} % Equations
    \usepackage{textcomp} % defines textquotesingle
    % Hack from http://tex.stackexchange.com/a/47451/13684:
    \AtBeginDocument{%
        \def\PYZsq{\textquotesingle}% Upright quotes in Pygmentized code
    }
    \usepackage{upquote} % Upright quotes for verbatim code
    \usepackage{eurosym} % defines \euro
    \usepackage[mathletters]{ucs} % Extended unicode (utf-8) support
    \usepackage[utf8x]{inputenc} % Allow utf-8 characters in the tex document
    \usepackage{fancyvrb} % verbatim replacement that allows latex
    \usepackage{grffile} % extends the file name processing of package graphics 
                         % to support a larger range 
    % The hyperref package gives us a pdf with properly built
    % internal navigation ('pdf bookmarks' for the table of contents,
    % internal cross-reference links, web links for URLs, etc.)
    \usepackage{hyperref}
    \usepackage{longtable} % longtable support required by pandoc >1.10
    \usepackage{booktabs}  % table support for pandoc > 1.12.2
    \usepackage[inline]{enumitem} % IRkernel/repr support (it uses the enumerate* environment)
    \usepackage[normalem]{ulem} % ulem is needed to support strikethroughs (\sout)
                                % normalem makes italics be italics, not underlines
    

    
    
    % Colors for the hyperref package
    \definecolor{urlcolor}{rgb}{0,.145,.698}
    \definecolor{linkcolor}{rgb}{.71,0.21,0.01}
    \definecolor{citecolor}{rgb}{.12,.54,.11}

    % ANSI colors
    \definecolor{ansi-black}{HTML}{3E424D}
    \definecolor{ansi-black-intense}{HTML}{282C36}
    \definecolor{ansi-red}{HTML}{E75C58}
    \definecolor{ansi-red-intense}{HTML}{B22B31}
    \definecolor{ansi-green}{HTML}{00A250}
    \definecolor{ansi-green-intense}{HTML}{007427}
    \definecolor{ansi-yellow}{HTML}{DDB62B}
    \definecolor{ansi-yellow-intense}{HTML}{B27D12}
    \definecolor{ansi-blue}{HTML}{208FFB}
    \definecolor{ansi-blue-intense}{HTML}{0065CA}
    \definecolor{ansi-magenta}{HTML}{D160C4}
    \definecolor{ansi-magenta-intense}{HTML}{A03196}
    \definecolor{ansi-cyan}{HTML}{60C6C8}
    \definecolor{ansi-cyan-intense}{HTML}{258F8F}
    \definecolor{ansi-white}{HTML}{C5C1B4}
    \definecolor{ansi-white-intense}{HTML}{A1A6B2}

    % commands and environments needed by pandoc snippets
    % extracted from the output of `pandoc -s`
    \providecommand{\tightlist}{%
      \setlength{\itemsep}{0pt}\setlength{\parskip}{0pt}}
    \DefineVerbatimEnvironment{Highlighting}{Verbatim}{commandchars=\\\{\}}
    % Add ',fontsize=\small' for more characters per line
    \newenvironment{Shaded}{}{}
    \newcommand{\KeywordTok}[1]{\textcolor[rgb]{0.00,0.44,0.13}{\textbf{{#1}}}}
    \newcommand{\DataTypeTok}[1]{\textcolor[rgb]{0.56,0.13,0.00}{{#1}}}
    \newcommand{\DecValTok}[1]{\textcolor[rgb]{0.25,0.63,0.44}{{#1}}}
    \newcommand{\BaseNTok}[1]{\textcolor[rgb]{0.25,0.63,0.44}{{#1}}}
    \newcommand{\FloatTok}[1]{\textcolor[rgb]{0.25,0.63,0.44}{{#1}}}
    \newcommand{\CharTok}[1]{\textcolor[rgb]{0.25,0.44,0.63}{{#1}}}
    \newcommand{\StringTok}[1]{\textcolor[rgb]{0.25,0.44,0.63}{{#1}}}
    \newcommand{\CommentTok}[1]{\textcolor[rgb]{0.38,0.63,0.69}{\textit{{#1}}}}
    \newcommand{\OtherTok}[1]{\textcolor[rgb]{0.00,0.44,0.13}{{#1}}}
    \newcommand{\AlertTok}[1]{\textcolor[rgb]{1.00,0.00,0.00}{\textbf{{#1}}}}
    \newcommand{\FunctionTok}[1]{\textcolor[rgb]{0.02,0.16,0.49}{{#1}}}
    \newcommand{\RegionMarkerTok}[1]{{#1}}
    \newcommand{\ErrorTok}[1]{\textcolor[rgb]{1.00,0.00,0.00}{\textbf{{#1}}}}
    \newcommand{\NormalTok}[1]{{#1}}
    
    % Additional commands for more recent versions of Pandoc
    \newcommand{\ConstantTok}[1]{\textcolor[rgb]{0.53,0.00,0.00}{{#1}}}
    \newcommand{\SpecialCharTok}[1]{\textcolor[rgb]{0.25,0.44,0.63}{{#1}}}
    \newcommand{\VerbatimStringTok}[1]{\textcolor[rgb]{0.25,0.44,0.63}{{#1}}}
    \newcommand{\SpecialStringTok}[1]{\textcolor[rgb]{0.73,0.40,0.53}{{#1}}}
    \newcommand{\ImportTok}[1]{{#1}}
    \newcommand{\DocumentationTok}[1]{\textcolor[rgb]{0.73,0.13,0.13}{\textit{{#1}}}}
    \newcommand{\AnnotationTok}[1]{\textcolor[rgb]{0.38,0.63,0.69}{\textbf{\textit{{#1}}}}}
    \newcommand{\CommentVarTok}[1]{\textcolor[rgb]{0.38,0.63,0.69}{\textbf{\textit{{#1}}}}}
    \newcommand{\VariableTok}[1]{\textcolor[rgb]{0.10,0.09,0.49}{{#1}}}
    \newcommand{\ControlFlowTok}[1]{\textcolor[rgb]{0.00,0.44,0.13}{\textbf{{#1}}}}
    \newcommand{\OperatorTok}[1]{\textcolor[rgb]{0.40,0.40,0.40}{{#1}}}
    \newcommand{\BuiltInTok}[1]{{#1}}
    \newcommand{\ExtensionTok}[1]{{#1}}
    \newcommand{\PreprocessorTok}[1]{\textcolor[rgb]{0.74,0.48,0.00}{{#1}}}
    \newcommand{\AttributeTok}[1]{\textcolor[rgb]{0.49,0.56,0.16}{{#1}}}
    \newcommand{\InformationTok}[1]{\textcolor[rgb]{0.38,0.63,0.69}{\textbf{\textit{{#1}}}}}
    \newcommand{\WarningTok}[1]{\textcolor[rgb]{0.38,0.63,0.69}{\textbf{\textit{{#1}}}}}
    
    
    % Define a nice break command that doesn't care if a line doesn't already
    % exist.
    \def\br{\hspace*{\fill} \\* }
    % Math Jax compatability definitions
    \def\gt{>}
    \def\lt{<}
    % Document parameters
    \title{Implementation}
    
    
    

    % Pygments definitions
    
\makeatletter
\def\PY@reset{\let\PY@it=\relax \let\PY@bf=\relax%
    \let\PY@ul=\relax \let\PY@tc=\relax%
    \let\PY@bc=\relax \let\PY@ff=\relax}
\def\PY@tok#1{\csname PY@tok@#1\endcsname}
\def\PY@toks#1+{\ifx\relax#1\empty\else%
    \PY@tok{#1}\expandafter\PY@toks\fi}
\def\PY@do#1{\PY@bc{\PY@tc{\PY@ul{%
    \PY@it{\PY@bf{\PY@ff{#1}}}}}}}
\def\PY#1#2{\PY@reset\PY@toks#1+\relax+\PY@do{#2}}

\expandafter\def\csname PY@tok@w\endcsname{\def\PY@tc##1{\textcolor[rgb]{0.73,0.73,0.73}{##1}}}
\expandafter\def\csname PY@tok@c\endcsname{\let\PY@it=\textit\def\PY@tc##1{\textcolor[rgb]{0.25,0.50,0.50}{##1}}}
\expandafter\def\csname PY@tok@cp\endcsname{\def\PY@tc##1{\textcolor[rgb]{0.74,0.48,0.00}{##1}}}
\expandafter\def\csname PY@tok@k\endcsname{\let\PY@bf=\textbf\def\PY@tc##1{\textcolor[rgb]{0.00,0.50,0.00}{##1}}}
\expandafter\def\csname PY@tok@kp\endcsname{\def\PY@tc##1{\textcolor[rgb]{0.00,0.50,0.00}{##1}}}
\expandafter\def\csname PY@tok@kt\endcsname{\def\PY@tc##1{\textcolor[rgb]{0.69,0.00,0.25}{##1}}}
\expandafter\def\csname PY@tok@o\endcsname{\def\PY@tc##1{\textcolor[rgb]{0.40,0.40,0.40}{##1}}}
\expandafter\def\csname PY@tok@ow\endcsname{\let\PY@bf=\textbf\def\PY@tc##1{\textcolor[rgb]{0.67,0.13,1.00}{##1}}}
\expandafter\def\csname PY@tok@nb\endcsname{\def\PY@tc##1{\textcolor[rgb]{0.00,0.50,0.00}{##1}}}
\expandafter\def\csname PY@tok@nf\endcsname{\def\PY@tc##1{\textcolor[rgb]{0.00,0.00,1.00}{##1}}}
\expandafter\def\csname PY@tok@nc\endcsname{\let\PY@bf=\textbf\def\PY@tc##1{\textcolor[rgb]{0.00,0.00,1.00}{##1}}}
\expandafter\def\csname PY@tok@nn\endcsname{\let\PY@bf=\textbf\def\PY@tc##1{\textcolor[rgb]{0.00,0.00,1.00}{##1}}}
\expandafter\def\csname PY@tok@ne\endcsname{\let\PY@bf=\textbf\def\PY@tc##1{\textcolor[rgb]{0.82,0.25,0.23}{##1}}}
\expandafter\def\csname PY@tok@nv\endcsname{\def\PY@tc##1{\textcolor[rgb]{0.10,0.09,0.49}{##1}}}
\expandafter\def\csname PY@tok@no\endcsname{\def\PY@tc##1{\textcolor[rgb]{0.53,0.00,0.00}{##1}}}
\expandafter\def\csname PY@tok@nl\endcsname{\def\PY@tc##1{\textcolor[rgb]{0.63,0.63,0.00}{##1}}}
\expandafter\def\csname PY@tok@ni\endcsname{\let\PY@bf=\textbf\def\PY@tc##1{\textcolor[rgb]{0.60,0.60,0.60}{##1}}}
\expandafter\def\csname PY@tok@na\endcsname{\def\PY@tc##1{\textcolor[rgb]{0.49,0.56,0.16}{##1}}}
\expandafter\def\csname PY@tok@nt\endcsname{\let\PY@bf=\textbf\def\PY@tc##1{\textcolor[rgb]{0.00,0.50,0.00}{##1}}}
\expandafter\def\csname PY@tok@nd\endcsname{\def\PY@tc##1{\textcolor[rgb]{0.67,0.13,1.00}{##1}}}
\expandafter\def\csname PY@tok@s\endcsname{\def\PY@tc##1{\textcolor[rgb]{0.73,0.13,0.13}{##1}}}
\expandafter\def\csname PY@tok@sd\endcsname{\let\PY@it=\textit\def\PY@tc##1{\textcolor[rgb]{0.73,0.13,0.13}{##1}}}
\expandafter\def\csname PY@tok@si\endcsname{\let\PY@bf=\textbf\def\PY@tc##1{\textcolor[rgb]{0.73,0.40,0.53}{##1}}}
\expandafter\def\csname PY@tok@se\endcsname{\let\PY@bf=\textbf\def\PY@tc##1{\textcolor[rgb]{0.73,0.40,0.13}{##1}}}
\expandafter\def\csname PY@tok@sr\endcsname{\def\PY@tc##1{\textcolor[rgb]{0.73,0.40,0.53}{##1}}}
\expandafter\def\csname PY@tok@ss\endcsname{\def\PY@tc##1{\textcolor[rgb]{0.10,0.09,0.49}{##1}}}
\expandafter\def\csname PY@tok@sx\endcsname{\def\PY@tc##1{\textcolor[rgb]{0.00,0.50,0.00}{##1}}}
\expandafter\def\csname PY@tok@m\endcsname{\def\PY@tc##1{\textcolor[rgb]{0.40,0.40,0.40}{##1}}}
\expandafter\def\csname PY@tok@gh\endcsname{\let\PY@bf=\textbf\def\PY@tc##1{\textcolor[rgb]{0.00,0.00,0.50}{##1}}}
\expandafter\def\csname PY@tok@gu\endcsname{\let\PY@bf=\textbf\def\PY@tc##1{\textcolor[rgb]{0.50,0.00,0.50}{##1}}}
\expandafter\def\csname PY@tok@gd\endcsname{\def\PY@tc##1{\textcolor[rgb]{0.63,0.00,0.00}{##1}}}
\expandafter\def\csname PY@tok@gi\endcsname{\def\PY@tc##1{\textcolor[rgb]{0.00,0.63,0.00}{##1}}}
\expandafter\def\csname PY@tok@gr\endcsname{\def\PY@tc##1{\textcolor[rgb]{1.00,0.00,0.00}{##1}}}
\expandafter\def\csname PY@tok@ge\endcsname{\let\PY@it=\textit}
\expandafter\def\csname PY@tok@gs\endcsname{\let\PY@bf=\textbf}
\expandafter\def\csname PY@tok@gp\endcsname{\let\PY@bf=\textbf\def\PY@tc##1{\textcolor[rgb]{0.00,0.00,0.50}{##1}}}
\expandafter\def\csname PY@tok@go\endcsname{\def\PY@tc##1{\textcolor[rgb]{0.53,0.53,0.53}{##1}}}
\expandafter\def\csname PY@tok@gt\endcsname{\def\PY@tc##1{\textcolor[rgb]{0.00,0.27,0.87}{##1}}}
\expandafter\def\csname PY@tok@err\endcsname{\def\PY@bc##1{\setlength{\fboxsep}{0pt}\fcolorbox[rgb]{1.00,0.00,0.00}{1,1,1}{\strut ##1}}}
\expandafter\def\csname PY@tok@kc\endcsname{\let\PY@bf=\textbf\def\PY@tc##1{\textcolor[rgb]{0.00,0.50,0.00}{##1}}}
\expandafter\def\csname PY@tok@kd\endcsname{\let\PY@bf=\textbf\def\PY@tc##1{\textcolor[rgb]{0.00,0.50,0.00}{##1}}}
\expandafter\def\csname PY@tok@kn\endcsname{\let\PY@bf=\textbf\def\PY@tc##1{\textcolor[rgb]{0.00,0.50,0.00}{##1}}}
\expandafter\def\csname PY@tok@kr\endcsname{\let\PY@bf=\textbf\def\PY@tc##1{\textcolor[rgb]{0.00,0.50,0.00}{##1}}}
\expandafter\def\csname PY@tok@bp\endcsname{\def\PY@tc##1{\textcolor[rgb]{0.00,0.50,0.00}{##1}}}
\expandafter\def\csname PY@tok@fm\endcsname{\def\PY@tc##1{\textcolor[rgb]{0.00,0.00,1.00}{##1}}}
\expandafter\def\csname PY@tok@vc\endcsname{\def\PY@tc##1{\textcolor[rgb]{0.10,0.09,0.49}{##1}}}
\expandafter\def\csname PY@tok@vg\endcsname{\def\PY@tc##1{\textcolor[rgb]{0.10,0.09,0.49}{##1}}}
\expandafter\def\csname PY@tok@vi\endcsname{\def\PY@tc##1{\textcolor[rgb]{0.10,0.09,0.49}{##1}}}
\expandafter\def\csname PY@tok@vm\endcsname{\def\PY@tc##1{\textcolor[rgb]{0.10,0.09,0.49}{##1}}}
\expandafter\def\csname PY@tok@sa\endcsname{\def\PY@tc##1{\textcolor[rgb]{0.73,0.13,0.13}{##1}}}
\expandafter\def\csname PY@tok@sb\endcsname{\def\PY@tc##1{\textcolor[rgb]{0.73,0.13,0.13}{##1}}}
\expandafter\def\csname PY@tok@sc\endcsname{\def\PY@tc##1{\textcolor[rgb]{0.73,0.13,0.13}{##1}}}
\expandafter\def\csname PY@tok@dl\endcsname{\def\PY@tc##1{\textcolor[rgb]{0.73,0.13,0.13}{##1}}}
\expandafter\def\csname PY@tok@s2\endcsname{\def\PY@tc##1{\textcolor[rgb]{0.73,0.13,0.13}{##1}}}
\expandafter\def\csname PY@tok@sh\endcsname{\def\PY@tc##1{\textcolor[rgb]{0.73,0.13,0.13}{##1}}}
\expandafter\def\csname PY@tok@s1\endcsname{\def\PY@tc##1{\textcolor[rgb]{0.73,0.13,0.13}{##1}}}
\expandafter\def\csname PY@tok@mb\endcsname{\def\PY@tc##1{\textcolor[rgb]{0.40,0.40,0.40}{##1}}}
\expandafter\def\csname PY@tok@mf\endcsname{\def\PY@tc##1{\textcolor[rgb]{0.40,0.40,0.40}{##1}}}
\expandafter\def\csname PY@tok@mh\endcsname{\def\PY@tc##1{\textcolor[rgb]{0.40,0.40,0.40}{##1}}}
\expandafter\def\csname PY@tok@mi\endcsname{\def\PY@tc##1{\textcolor[rgb]{0.40,0.40,0.40}{##1}}}
\expandafter\def\csname PY@tok@il\endcsname{\def\PY@tc##1{\textcolor[rgb]{0.40,0.40,0.40}{##1}}}
\expandafter\def\csname PY@tok@mo\endcsname{\def\PY@tc##1{\textcolor[rgb]{0.40,0.40,0.40}{##1}}}
\expandafter\def\csname PY@tok@ch\endcsname{\let\PY@it=\textit\def\PY@tc##1{\textcolor[rgb]{0.25,0.50,0.50}{##1}}}
\expandafter\def\csname PY@tok@cm\endcsname{\let\PY@it=\textit\def\PY@tc##1{\textcolor[rgb]{0.25,0.50,0.50}{##1}}}
\expandafter\def\csname PY@tok@cpf\endcsname{\let\PY@it=\textit\def\PY@tc##1{\textcolor[rgb]{0.25,0.50,0.50}{##1}}}
\expandafter\def\csname PY@tok@c1\endcsname{\let\PY@it=\textit\def\PY@tc##1{\textcolor[rgb]{0.25,0.50,0.50}{##1}}}
\expandafter\def\csname PY@tok@cs\endcsname{\let\PY@it=\textit\def\PY@tc##1{\textcolor[rgb]{0.25,0.50,0.50}{##1}}}

\def\PYZbs{\char`\\}
\def\PYZus{\char`\_}
\def\PYZob{\char`\{}
\def\PYZcb{\char`\}}
\def\PYZca{\char`\^}
\def\PYZam{\char`\&}
\def\PYZlt{\char`\<}
\def\PYZgt{\char`\>}
\def\PYZsh{\char`\#}
\def\PYZpc{\char`\%}
\def\PYZdl{\char`\$}
\def\PYZhy{\char`\-}
\def\PYZsq{\char`\'}
\def\PYZdq{\char`\"}
\def\PYZti{\char`\~}
% for compatibility with earlier versions
\def\PYZat{@}
\def\PYZlb{[}
\def\PYZrb{]}
\makeatother


    % Exact colors from NB
    \definecolor{incolor}{rgb}{0.0, 0.0, 0.5}
    \definecolor{outcolor}{rgb}{0.545, 0.0, 0.0}



    
    % Prevent overflowing lines due to hard-to-break entities
    \sloppy 
    % Setup hyperref package
    \hypersetup{
      breaklinks=true,  % so long urls are correctly broken across lines
      colorlinks=true,
      urlcolor=urlcolor,
      linkcolor=linkcolor,
      citecolor=citecolor,
      }
    % Slightly bigger margins than the latex defaults
    
    \geometry{verbose,tmargin=1in,bmargin=1in,lmargin=1in,rmargin=1in}
    
    

    \begin{document}
    
    
    \maketitle
    
    

    
    \hypertarget{implementation}{%
\section{Implementation}\label{implementation}}

\hypertarget{using-probabilistic-graphical-models-to-forecast-the-price-of-crude-oil}{%
\paragraph{(Using Probabilistic Graphical Models to forecast the price
of crude
oil)}\label{using-probabilistic-graphical-models-to-forecast-the-price-of-crude-oil}}

The document contains the \textbf{implementation} of the oil trading
system using graphical models.

\begin{itemize}
\tightlist
\item
  Section 1 is dedicated to retrieving data from the
  \textbf{EIA} and \textbf{FRED}, \textbf{preprocessing} the data,and
  creating the \textbf{training}, \textbf{validation}, and \textbf{test}
  datasets.
\item
  Section 2 is dedicated to implementing a \textbf{regime
  detection model} using \textbf{Hidden Markov Models} to identify bull,
  bear, and stagnant regimes.
\item
  Section 3 is dedicated to learning the \textbf{macroeconomic
  structure} of the oil markets by \textbf{learning the belief network}
  using \textbf{hill-climbing} structural learning.
\item
  Section 4 is dedicated to \textbf{testing} the constructed
  model by simulating trades and taking positions based on those trades.
\end{itemize}

It is recommended to take look at the thesis documentation to understand
the basis on which we selected the macroeconomic economic data from the
EIA and FRED, and the theoretical context of the graphical models being
employed in our model.

We would be using a number of Python packages, such as
\href{http://pgmpy.org/}{pgmpy} and
\href{https://github.com/lopatovsky/HMMs}{hmms} throughout the notebook
and it is highly recommended that we take a deeper look at them as only
a specific and relevant functionality of those packages have been used
in our model.

    \hypertarget{data-preprocessing}{%
\subsection{Data preprocessing}\label{data-preprocessing}}

    Data preprocessing plays an important role in Machine Learning. Our data
preprocessing has four main steps; data retrieval, data cleaning, data
transformation, and data discretisation.

\hypertarget{data-retrieval}{%
\subsubsection{Data retrieval}\label{data-retrieval}}

The first step of constructing our model is to retrieve the data from
the open-data facilities. We have selected the \textbf{EIA} and
\textbf{FRED} as our primary data sources. Unfortunately, both these
open-data facilities do not provide Python packages to neatly retrieve
data in Python, so we will have to resort to using third-party APIs. For
the EIA, we are using
\href{https://github.com/mra1385/EIA-python/}{EIA-python}, and for the
FRED we are using \href{https://github.com/mortada/fredapi}{fredapi}.

Before beginning, we would be first be importing pandas and numpy, as
they are highly required in the entire data preprocessing section.

    \begin{Verbatim}[commandchars=\\\{\}]
{\color{incolor}In [{\color{incolor}1}]:} \PY{k+kn}{import} \PY{n+nn}{pandas} \PY{k}{as} \PY{n+nn}{pd} \PY{c+c1}{\PYZsh{} pandas is an open source, data structures and data analysis tool in Python}
        \PY{k+kn}{import} \PY{n+nn}{numpy} \PY{k}{as} \PY{n+nn}{np}  \PY{c+c1}{\PYZsh{} numpy is a Python library supporting high\PYZhy{}level mathematical functions on large multidimensional arrays}
\end{Verbatim}


    We would now be retrieving data, beginning with the EIA. Before
retrieving data from the EIA, we have to register with EIA's
\href{https://www.eia.gov/opendata/}{open-data facility}, in return of
which we shall recieve an API key, which is used as a passphrase to
access data from the EIA's datacenter.

    \begin{Verbatim}[commandchars=\\\{\}]
{\color{incolor}In [{\color{incolor}2}]:} \PY{k+kn}{import} \PY{n+nn}{eia} \PY{c+c1}{\PYZsh{} Importing the library}
        \PY{n}{eia\PYZus{}key} \PY{o}{=} \PY{l+s+s2}{\PYZdq{}}\PY{l+s+s2}{265d1f2178aaab3ceec3d364d9cc1d11}\PY{l+s+s2}{\PYZdq{}}\PY{p}{;} \PY{c+c1}{\PYZsh{} the API key we recieved from EIA}
        \PY{n}{eia\PYZus{}api} \PY{o}{=} \PY{n}{eia}\PY{o}{.}\PY{n}{API}\PY{p}{(}\PY{n}{eia\PYZus{}key}\PY{p}{)}\PY{p}{;} \PY{c+c1}{\PYZsh{} Initiates a session with the EIA datacenter to recieve datasets}
\end{Verbatim}


    Now, we shall be making a request to retrieve data from the EIA as a
pandas dataframe. EIA provides a 3,872 Short-Term Energy Outlook (STEO)
datasets, with short-term (2-year) forecasts of each dataset. These
datasets can be searched in EIAs
\href{https://www.eia.gov/opendata/qb.php}{query browse} facility, which
also offers a catalogue of different datasets sorted by relevance. Just
as an example to demonstrate, we would be retrieving the \textbf{Crude
Oil Exports, Monthly}, which has a Series ID `\textbf{TOTAL.COEXPUS.M}'.

    \begin{Verbatim}[commandchars=\\\{\}]
{\color{incolor}In [{\color{incolor}3}]:} \PY{n}{eia\PYZus{}data} \PY{o}{=} \PY{n}{pd}\PY{o}{.}\PY{n}{DataFrame}\PY{p}{(}\PY{n}{eia\PYZus{}api}\PY{o}{.}\PY{n}{data\PYZus{}by\PYZus{}series}\PY{p}{(}\PY{n}{series}\PY{o}{=}\PY{l+s+s1}{\PYZsq{}}\PY{l+s+s1}{TOTAL.COEXPUS.M}\PY{l+s+s1}{\PYZsq{}}\PY{p}{)}\PY{p}{)}\PY{p}{;} \PY{c+c1}{\PYZsh{} Convert to pandas dataframe}
\end{Verbatim}


    \hypertarget{data-cleaning}{%
\subsubsection{Data Cleaning}\label{data-cleaning}}

Taking a look at the dataframe, we can observe some evident
inconsistencies.

Firstly, the dataframe provided by the EIA is not of the standard format
\textbf{datetime}, which pandas indexing supports and provides extensive
facility to. We would be writing a function which makes the index a
\textbf{datetime} object so that we can convert the dataframe to a
\textbf{datetime}-index dataframe for more compatibility with pandas,
hmms, and pgmpy.

    \begin{Verbatim}[commandchars=\\\{\}]
{\color{incolor}In [{\color{incolor}4}]:} \PY{k+kn}{import} \PY{n+nn}{datetime} \PY{c+c1}{\PYZsh{} Using the datetime library}
        
        \PY{k}{def} \PY{n+nf}{convert\PYZus{}to\PYZus{}datetime}\PY{p}{(}\PY{n+nb}{input}\PY{p}{)}\PY{p}{:}
                    \PY{k}{return} \PY{n}{datetime}\PY{o}{.}\PY{n}{datetime}\PY{o}{.}\PY{n}{strptime}\PY{p}{(}\PY{n+nb}{input}\PY{p}{[}\PY{p}{:}\PY{l+m+mi}{9}\PY{p}{]}\PY{p}{,} \PY{l+s+s2}{\PYZdq{}}\PY{l+s+s2}{\PYZpc{}}\PY{l+s+s2}{Y }\PY{l+s+s2}{\PYZpc{}}\PY{l+s+s2}{m }\PY{l+s+s2}{\PYZdq{}}\PY{p}{)}\PY{o}{.}\PY{n}{date}\PY{p}{(}\PY{p}{)}\PY{p}{;}
        
        \PY{n}{eia\PYZus{}data}\PY{o}{.}\PY{n}{index} \PY{o}{=} \PY{n}{eia\PYZus{}data}\PY{o}{.}\PY{n}{index}\PY{o}{.}\PY{n}{map}\PY{p}{(}\PY{n}{convert\PYZus{}to\PYZus{}datetime}\PY{p}{)}\PY{p}{;} \PY{c+c1}{\PYZsh{} Apply to entire index}
        \PY{n}{eia\PYZus{}data}\PY{o}{.}\PY{n}{index} \PY{o}{=} \PY{n}{pd}\PY{o}{.}\PY{n}{to\PYZus{}datetime}\PY{p}{(}\PY{n}{eia\PYZus{}data}\PY{o}{.}\PY{n}{index}\PY{p}{)}\PY{p}{;} \PY{c+c1}{\PYZsh{} Convert dataframe index to datetime64[ns] index}
        \PY{n}{eia\PYZus{}data}\PY{o}{.}\PY{n}{columns} \PY{o}{=} \PY{p}{[}\PY{l+s+s1}{\PYZsq{}}\PY{l+s+s1}{TOTAL.COEXPUS.M}\PY{l+s+s1}{\PYZsq{}}\PY{p}{]}\PY{p}{;} \PY{c+c1}{\PYZsh{} pgmpy stores the column names as the variable name}
\end{Verbatim}


    The second issue are \textbf{holes} in the data i.e.~rows marked by a
`-' (a single dash). We would be replacing these dashes by
\textbf{np.nan} so that we can use pandas to fill in the holes. Usually
the prevalance of these holes is very rare, but just to be on the safe
side to ensure we can possibly download every dataset.

    \begin{Verbatim}[commandchars=\\\{\}]
{\color{incolor}In [{\color{incolor}5}]:} \PY{n}{eia\PYZus{}data}\PY{o}{.}\PY{n}{replace}\PY{p}{(}\PY{l+s+s1}{\PYZsq{}}\PY{l+s+s1}{\PYZhy{}}\PY{l+s+s1}{\PYZsq{}}\PY{p}{,} \PY{n}{np}\PY{o}{.}\PY{n}{nan}\PY{p}{,} \PY{n}{regex}\PY{o}{=}\PY{k+kc}{True}\PY{p}{,} \PY{n}{inplace}\PY{o}{=}\PY{k+kc}{True}\PY{p}{)}\PY{p}{;} \PY{c+c1}{\PYZsh{} Replace the \PYZsq{}\PYZhy{}\PYZsq{} with np.nan}
        \PY{n}{eia\PYZus{}data}\PY{o}{.}\PY{n}{fillna}\PY{p}{(}\PY{n}{method}\PY{o}{=}\PY{l+s+s1}{\PYZsq{}}\PY{l+s+s1}{bfill}\PY{l+s+s1}{\PYZsq{}}\PY{p}{,} \PY{n}{inplace}\PY{o}{=}\PY{k+kc}{True}\PY{p}{)}\PY{p}{;} \PY{c+c1}{\PYZsh{} Backward fill the holes, by filling them with the data infront.}
\end{Verbatim}


    Together, we can create a function carrying out the entire process so
that we can easily clean EIA data in one step.

    \begin{Verbatim}[commandchars=\\\{\}]
{\color{incolor}In [{\color{incolor}6}]:} \PY{k}{def} \PY{n+nf}{clean\PYZus{}EIA}\PY{p}{(}\PY{n}{data}\PY{p}{)}\PY{p}{:}
                    \PY{n}{data}\PY{o}{.}\PY{n}{replace}\PY{p}{(}\PY{l+s+s1}{\PYZsq{}}\PY{l+s+s1}{\PYZhy{}}\PY{l+s+s1}{\PYZsq{}}\PY{p}{,} \PY{n}{np}\PY{o}{.}\PY{n}{nan}\PY{p}{,} \PY{n}{regex}\PY{o}{=}\PY{k+kc}{True}\PY{p}{,} \PY{n}{inplace}\PY{o}{=}\PY{k+kc}{True}\PY{p}{)}\PY{p}{;}
                    \PY{n}{data}\PY{o}{.}\PY{n}{fillna}\PY{p}{(}\PY{n}{method}\PY{o}{=}\PY{l+s+s1}{\PYZsq{}}\PY{l+s+s1}{bfill}\PY{l+s+s1}{\PYZsq{}}\PY{p}{,} \PY{n}{inplace}\PY{o}{=}\PY{k+kc}{True}\PY{p}{)}\PY{p}{;}
                    
                    \PY{n}{data}\PY{o}{.}\PY{n}{index} \PY{o}{=} \PY{n}{data}\PY{o}{.}\PY{n}{index}\PY{o}{.}\PY{n}{map}\PY{p}{(}\PY{n}{convert\PYZus{}to\PYZus{}datetime}\PY{p}{)}\PY{p}{;}
                    \PY{n}{data}\PY{o}{.}\PY{n}{index} \PY{o}{=} \PY{n}{pd}\PY{o}{.}\PY{n}{to\PYZus{}datetime}\PY{p}{(}\PY{n}{data}\PY{o}{.}\PY{n}{index}\PY{p}{)}\PY{p}{;}
\end{Verbatim}


    The dataframe is now a time series dataframe which could be plotted as a
time-series dataframe.

    \begin{Verbatim}[commandchars=\\\{\}]
{\color{incolor}In [{\color{incolor}7}]:} \PY{k+kn}{import} \PY{n+nn}{matplotlib}\PY{n+nn}{.}\PY{n+nn}{pyplot} \PY{k}{as} \PY{n+nn}{plt}
        
        \PY{n}{fig}\PY{p}{,} \PY{n}{ax} \PY{o}{=} \PY{n}{plt}\PY{o}{.}\PY{n}{subplots}\PY{p}{(}\PY{n}{figsize}\PY{o}{=}\PY{p}{(}\PY{l+m+mi}{20}\PY{p}{,}\PY{l+m+mi}{6}\PY{p}{)}\PY{p}{)}\PY{p}{;}
        \PY{n}{ax}\PY{o}{.}\PY{n}{plot}\PY{p}{(}\PY{n}{eia\PYZus{}data}\PY{p}{)}\PY{p}{;}
\end{Verbatim}


    Now, we shall be taking a look at the FRED data. Similar to
\href{https://github.com/mra1385/EIA-python/}{EIA-python}, the
\href{https://github.com/mortada/fredapi}{fredapi} requires us to
register with \href{https://research.stlouisfed.org/docs/api/fred/}{FRED
API} so that we can access data. We would download the \textbf{Spot
Crude Oil Price: West Texas Intermediate}, having the Series ID
`\textbf{WTISPLC}.'

    \begin{Verbatim}[commandchars=\\\{\}]
{\color{incolor}In [{\color{incolor}8}]:} \PY{k+kn}{from} \PY{n+nn}{fredapi} \PY{k}{import} \PY{n}{Fred}
        
        \PY{n}{fred\PYZus{}key} \PY{o}{=} \PY{l+s+s2}{\PYZdq{}}\PY{l+s+s2}{029c72315e9ec4eaf3e679ec3f6a2cb3}\PY{l+s+s2}{\PYZdq{}}\PY{p}{;} \PY{c+c1}{\PYZsh{} FRED API key}
        \PY{n}{fred} \PY{o}{=} \PY{n}{Fred}\PY{p}{(}\PY{n}{api\PYZus{}key}\PY{o}{=}\PY{n}{fred\PYZus{}key}\PY{p}{)}\PY{p}{;} \PY{c+c1}{\PYZsh{} Initiates a session with the FRED datacenter to recieve datasets}
        
        \PY{n}{fred\PYZus{}data} \PY{o}{=} \PY{n}{pd}\PY{o}{.}\PY{n}{DataFrame}\PY{p}{(}\PY{n}{fred}\PY{o}{.}\PY{n}{get\PYZus{}series}\PY{p}{(}\PY{l+s+s1}{\PYZsq{}}\PY{l+s+s1}{WTISPLC}\PY{l+s+s1}{\PYZsq{}}\PY{p}{)}\PY{p}{,} \PY{n}{columns}\PY{o}{=}\PY{p}{[}\PY{l+s+s1}{\PYZsq{}}\PY{l+s+s1}{WTISPLC}\PY{l+s+s1}{\PYZsq{}}\PY{p}{]}\PY{p}{)}\PY{p}{;} \PY{c+c1}{\PYZsh{} Retrieve data from FRED API}
\end{Verbatim}


    It is evident that the FRED, though still being a government
organization, has `ready-to-use' / `plug'n play' data of useable quality
compared to the EIA. Fortuntely, we will not be having to clean data
obtained from the FRED.

\hypertarget{constructing-the-training-validation-and-testing-datasets-data-transformation}{%
\subsubsection{Constructing the training, validation and testing
datasets (Data
transformation)}\label{constructing-the-training-validation-and-testing-datasets-data-transformation}}

As mentioned in the thesis, we need to divide our data in three
portions: the training dataset, the validation dataset, and the training
dataset. Given that we would be using a number of datasets from the FRED
and the EIA, we would have to amalgamate these datasets into one
dataframe and then slice the dataframe accordingly.

The train, validation, and test datasets are to be observed with a ratio
of \(80:10:10\), which is a popular ettiquette

The choice of datasets has been described in the thesis.

    \begin{Verbatim}[commandchars=\\\{\}]
{\color{incolor}In [{\color{incolor}9}]:} \PY{c+c1}{\PYZsh{} Dataset series ID from the EIA}
        
        \PY{n}{datasets\PYZus{}eia}  \PY{o}{=} \PY{p}{[}
            
                                \PY{l+s+s1}{\PYZsq{}}\PY{l+s+s1}{STEO.RGDPQ\PYZus{}NONOECD.M}\PY{l+s+s1}{\PYZsq{}}\PY{p}{,}
                                \PY{l+s+s1}{\PYZsq{}}\PY{l+s+s1}{STEO.RGDPQ\PYZus{}OECD.M}\PY{l+s+s1}{\PYZsq{}}\PY{p}{,}
            
                                \PY{l+s+s1}{\PYZsq{}}\PY{l+s+s1}{STEO.PAPR\PYZus{}NONOPEC.M}\PY{l+s+s1}{\PYZsq{}}\PY{p}{,}
                                \PY{l+s+s1}{\PYZsq{}}\PY{l+s+s1}{STEO.PAPR\PYZus{}OPEC.M}\PY{l+s+s1}{\PYZsq{}}\PY{p}{,}
            
                                \PY{l+s+s1}{\PYZsq{}}\PY{l+s+s1}{STEO.PATC\PYZus{}OECD.M}\PY{l+s+s1}{\PYZsq{}}\PY{p}{,}
                                \PY{l+s+s1}{\PYZsq{}}\PY{l+s+s1}{STEO.PATC\PYZus{}NON\PYZus{}OECD.M}\PY{l+s+s1}{\PYZsq{}}\PY{p}{,}
                        
            
                                \PY{l+s+s1}{\PYZsq{}}\PY{l+s+s1}{STEO.COPRPUS.M}\PY{l+s+s1}{\PYZsq{}}\PY{p}{,}
                                \PY{l+s+s1}{\PYZsq{}}\PY{l+s+s1}{STEO.CORIPUS.M}\PY{l+s+s1}{\PYZsq{}}\PY{p}{,}
                                \PY{l+s+s1}{\PYZsq{}}\PY{l+s+s1}{PET.MCRIMXX2.M}\PY{l+s+s1}{\PYZsq{}}\PY{p}{,}
                                
                                \PY{l+s+s1}{\PYZsq{}}\PY{l+s+s1}{STEO.FOREX\PYZus{}WORLD.M}\PY{l+s+s1}{\PYZsq{}}\PY{p}{,}
            
                                \PY{l+s+s1}{\PYZsq{}}\PY{l+s+s1}{STEO.PASC\PYZus{}OECD\PYZus{}T3.M}\PY{l+s+s1}{\PYZsq{}}\PY{p}{,}
            
                                \PY{l+s+s1}{\PYZsq{}}\PY{l+s+s1}{STEO.COPS\PYZus{}OPEC.M}\PY{l+s+s1}{\PYZsq{}}\PY{p}{,}
                                \PY{l+s+s1}{\PYZsq{}}\PY{l+s+s1}{STEO.COPC\PYZus{}OPEC.M}\PY{l+s+s1}{\PYZsq{}}\PY{p}{,}
            
                                \PY{l+s+s1}{\PYZsq{}}\PY{l+s+s1}{STEO.T3\PYZus{}STCHANGE\PYZus{}OOECD.M}\PY{l+s+s1}{\PYZsq{}}\PY{p}{,}
                                \PY{l+s+s1}{\PYZsq{}}\PY{l+s+s1}{STEO.T3\PYZus{}STCHANGE\PYZus{}NOECD.M}\PY{l+s+s1}{\PYZsq{}}\PY{p}{,}
                        \PY{p}{]}\PY{p}{;}
        
        \PY{c+c1}{\PYZsh{} Dataset series ID from the FRED}
        
        \PY{n}{datasets\PYZus{}fred} \PY{o}{=} \PY{p}{[}
                                \PY{l+s+s1}{\PYZsq{}}\PY{l+s+s1}{CPIENGSL}\PY{l+s+s1}{\PYZsq{}}\PY{p}{,}
                                \PY{l+s+s1}{\PYZsq{}}\PY{l+s+s1}{CAPG211S}\PY{l+s+s1}{\PYZsq{}}\PY{p}{,}
                                \PY{l+s+s1}{\PYZsq{}}\PY{l+s+s1}{CAPUTLG211S}\PY{l+s+s1}{\PYZsq{}}\PY{p}{,}
                                \PY{l+s+s1}{\PYZsq{}}\PY{l+s+s1}{IPG211S}\PY{l+s+s1}{\PYZsq{}}\PY{p}{,}
                                \PY{l+s+s1}{\PYZsq{}}\PY{l+s+s1}{IPG211111CN}\PY{l+s+s1}{\PYZsq{}}\PY{p}{,}
                                \PY{l+s+s1}{\PYZsq{}}\PY{l+s+s1}{INDPRO}\PY{l+s+s1}{\PYZsq{}}\PY{p}{,}
                                \PY{l+s+s1}{\PYZsq{}}\PY{l+s+s1}{IPN213111N}\PY{l+s+s1}{\PYZsq{}}\PY{p}{,}
                                \PY{l+s+s1}{\PYZsq{}}\PY{l+s+s1}{PCU211211}\PY{l+s+s1}{\PYZsq{}}\PY{p}{,}
                        
                        \PY{p}{]}\PY{p}{;}
\end{Verbatim}


    To construct the training, validation, and testing datasets, we need to
first \textbf{concatenate} the datasets into one dataframe, and then
slice it.

    \begin{Verbatim}[commandchars=\\\{\}]
{\color{incolor}In [{\color{incolor}10}]:} \PY{n}{data\PYZus{}merge} \PY{o}{=} \PY{p}{[}\PY{p}{]}\PY{p}{;} \PY{c+c1}{\PYZsh{} List of dataframes to be concatenated}
         
         \PY{c+c1}{\PYZsh{} Adding EIA datasets }
         
         \PY{k}{for} \PY{n}{series\PYZus{}id} \PY{o+ow}{in} \PY{n}{datasets\PYZus{}eia}\PY{p}{:}
                 \PY{n}{df} \PY{o}{=} \PY{n}{pd}\PY{o}{.}\PY{n}{DataFrame}\PY{p}{(}\PY{n}{eia\PYZus{}api}\PY{o}{.}\PY{n}{data\PYZus{}by\PYZus{}series}\PY{p}{(}\PY{n}{series}\PY{o}{=}\PY{n}{series\PYZus{}id}\PY{p}{)}\PY{p}{)}\PY{p}{;}
                 \PY{n}{clean\PYZus{}EIA}\PY{p}{(}\PY{n}{df}\PY{p}{)}\PY{p}{;}
                 \PY{n}{df}\PY{o}{.}\PY{n}{columns} \PY{o}{=} \PY{p}{[}\PY{n}{series\PYZus{}id}\PY{p}{]}\PY{p}{;}   
                 \PY{n}{data\PYZus{}merge}\PY{o}{.}\PY{n}{append}\PY{p}{(}\PY{n}{df}\PY{p}{)}\PY{p}{;}
         
         \PY{c+c1}{\PYZsh{} Adding FRED datasets }
         
         \PY{k}{for} \PY{n}{series\PYZus{}id} \PY{o+ow}{in} \PY{n}{datasets\PYZus{}fred}\PY{p}{:}
                 \PY{n}{df} \PY{o}{=} \PY{n}{pd}\PY{o}{.}\PY{n}{DataFrame}\PY{p}{(}\PY{n}{fred}\PY{o}{.}\PY{n}{get\PYZus{}series}\PY{p}{(}\PY{n}{series\PYZus{}id}\PY{p}{)}\PY{p}{,} \PY{n}{columns}\PY{o}{=}\PY{p}{[}\PY{n}{series\PYZus{}id}\PY{p}{]}\PY{p}{)}\PY{p}{;}
                 \PY{n}{data\PYZus{}merge}\PY{o}{.}\PY{n}{append}\PY{p}{(}\PY{n}{df}\PY{p}{)}\PY{p}{;}
\end{Verbatim}


    We have to create two additional columns; one which has the current
crude oil price, and the other for the price of crude oil next month
(forecast). This will be used to forecast the price of oil and hence
allow us to make buy/sell decisions based on that forecast.

    \begin{Verbatim}[commandchars=\\\{\}]
{\color{incolor}In [{\color{incolor}11}]:} \PY{n}{datasets} \PY{o}{=} \PY{n}{datasets\PYZus{}eia} \PY{o}{+} \PY{n}{datasets\PYZus{}fred} \PY{o}{+} \PY{p}{[}\PY{l+s+s1}{\PYZsq{}}\PY{l+s+s1}{WTISPLC}\PY{l+s+s1}{\PYZsq{}}\PY{p}{,} \PY{l+s+s1}{\PYZsq{}}\PY{l+s+s1}{forecast}\PY{l+s+s1}{\PYZsq{}}\PY{p}{]}\PY{p}{;}
         
         \PY{n}{current} \PY{o}{=}  \PY{n}{pd}\PY{o}{.}\PY{n}{DataFrame}\PY{p}{(}\PY{n}{fred}\PY{o}{.}\PY{n}{get\PYZus{}series}\PY{p}{(}\PY{l+s+s1}{\PYZsq{}}\PY{l+s+s1}{WTISPLC}\PY{l+s+s1}{\PYZsq{}}\PY{p}{)}\PY{p}{,} \PY{n}{columns}\PY{o}{=}\PY{p}{[}\PY{l+s+s1}{\PYZsq{}}\PY{l+s+s1}{WTISPLC}\PY{l+s+s1}{\PYZsq{}}\PY{p}{]}\PY{p}{)}\PY{p}{;}
         \PY{n}{forecast} \PY{o}{=} \PY{n}{pd}\PY{o}{.}\PY{n}{DataFrame}\PY{p}{(}\PY{n}{fred}\PY{o}{.}\PY{n}{get\PYZus{}series}\PY{p}{(}\PY{l+s+s1}{\PYZsq{}}\PY{l+s+s1}{WTISPLC}\PY{l+s+s1}{\PYZsq{}}\PY{p}{)}\PY{o}{.}\PY{n}{shift}\PY{p}{(}\PY{o}{\PYZhy{}}\PY{l+m+mi}{1}\PY{p}{)}\PY{p}{,} \PY{n}{columns}\PY{o}{=}\PY{p}{[}\PY{l+s+s1}{\PYZsq{}}\PY{l+s+s1}{forecast}\PY{l+s+s1}{\PYZsq{}}\PY{p}{]}\PY{p}{)}\PY{p}{;}
         
         \PY{n}{data\PYZus{}merge}\PY{o}{.}\PY{n}{append}\PY{p}{(}\PY{n}{current}\PY{p}{)}\PY{p}{;}
         \PY{n}{data\PYZus{}merge}\PY{o}{.}\PY{n}{append}\PY{p}{(}\PY{n}{forecast}\PY{p}{)}\PY{p}{;}
\end{Verbatim}


    We have to amalgamate all datasets together in a single dataframe,
therefore we would use the pandas \textbf{concatenate} function. This
would allow us to find the intersection of the date intervals of all
dataframes and construct a single dataframe on a common time interval.

    \begin{Verbatim}[commandchars=\\\{\}]
{\color{incolor}In [{\color{incolor}12}]:} \PY{n}{data} \PY{o}{=} \PY{n}{pd}\PY{o}{.}\PY{n}{concat}\PY{p}{(}\PY{n}{data\PYZus{}merge}\PY{p}{,} \PY{n}{axis}\PY{o}{=}\PY{l+m+mi}{1}\PY{p}{,} \PY{n}{join}\PY{o}{=}\PY{l+s+s1}{\PYZsq{}}\PY{l+s+s1}{inner}\PY{l+s+s1}{\PYZsq{}}\PY{p}{)}\PY{p}{;}
\end{Verbatim}


    Slicing our dataframe in train, validation, and testing datasets,

    \begin{Verbatim}[commandchars=\\\{\}]
{\color{incolor}In [{\color{incolor}13}]:} \PY{n}{train\PYZus{}data} \PY{o}{=} \PY{n}{data}\PY{p}{[}\PY{p}{:} \PY{n+nb}{int}\PY{p}{(}\PY{n}{data}\PY{o}{.}\PY{n}{shape}\PY{p}{[}\PY{l+m+mi}{0}\PY{p}{]} \PY{o}{*} \PY{l+m+mf}{0.80}\PY{p}{)}\PY{p}{]}\PY{p}{;}
         \PY{n}{vald\PYZus{}data} \PY{o}{=} \PY{n}{data}\PY{p}{[}\PY{n+nb}{int}\PY{p}{(}\PY{l+m+mf}{0.80} \PY{o}{*} \PY{n}{data}\PY{o}{.}\PY{n}{shape}\PY{p}{[}\PY{l+m+mi}{0}\PY{p}{]}\PY{p}{)} \PY{p}{:} \PY{n+nb}{int}\PY{p}{(}\PY{l+m+mf}{0.90} \PY{o}{*} \PY{n}{data}\PY{o}{.}\PY{n}{shape}\PY{p}{[}\PY{l+m+mi}{0}\PY{p}{]}\PY{p}{)}\PY{p}{]}\PY{p}{;}
         \PY{n}{test\PYZus{}data} \PY{o}{=} \PY{n}{data}\PY{p}{[}\PY{n+nb}{int}\PY{p}{(}\PY{l+m+mf}{0.90}\PY{o}{*} \PY{n}{data}\PY{o}{.}\PY{n}{shape}\PY{p}{[}\PY{l+m+mi}{0}\PY{p}{]}\PY{p}{)} \PY{p}{:} \PY{n+nb}{int}\PY{p}{(}\PY{n}{data}\PY{o}{.}\PY{n}{shape}\PY{p}{[}\PY{l+m+mi}{0}\PY{p}{]}\PY{p}{)}\PY{p}{]}\PY{p}{;}
\end{Verbatim}


    \hypertarget{data-discretisation}{%
\subsubsection{Data Discretisation}\label{data-discretisation}}

The data we have collected is non-categorical data; it is unlabelled and
continuous. Belief networks have variables, each having discrete
\textbf{states}, and therefore we have to reduce our data from prices to
a set of states, such as bull, bear, and stagnant markets. In order to
detect these (hidden) states, we have to use graphical models called
\textbf{Hidden Markov Models}. The process of detecting hidden states in
time-series data is called \textbf{Regime Detection}.

    \hypertarget{regime-detection}{%
\subsection{Regime Detection}\label{regime-detection}}

We would be using Python library called
\href{https://github.com/lopatovsky/HMMs}{hmms} for implementing the
\textbf{Hidden Markov Models}.

A \textbf{Hidden Markov Model} is a 5-tuple \((Q, \sum, \Pi, A, B)\),
where \(Q = \{q_{1}, \cdots, q_{N}\}\) is a finite set of
\(\mathcal{N}\) states, \(\sum = \{s_1, \cdots, s_{N}\}\) is the set of
\(\mathcal{M}\) possible symbols (emissions) in the language,
\(\Pi = \{\pi_{i}\}\) is the initial probability vector,
\(A = \{a_{ij}\}\) is the state transition probability matrix, and
\(B = \{b_i(v_k)\}\) is the emission probability matrix. The HMM can be
denoted by \(\lambda = (\Pi, A, B)\).

For detecting regimes in time-series data, we would be using
\textbf{Hidden Markov Models}, with the difference between consecutive
months being the symbols \(\sum\) (1 - increase / 0 - decrease), the
hidden states, \(Q\) being the \textbf{bull}, \textbf{bear},
\textbf{stagnant} market regimes.

Let us use `\textbf{WTISPL}' (Spot Crude Oil Price: West Texas
Intermediate) of and try to identify regimes in the time series.

    \begin{Verbatim}[commandchars=\\\{\}]
{\color{incolor}In [{\color{incolor}14}]:} \PY{k+kn}{import} \PY{n+nn}{hmms}
\end{Verbatim}


    \begin{Verbatim}[commandchars=\\\{\}]
{\color{incolor}In [{\color{incolor}15}]:} \PY{n}{price} \PY{o}{=} \PY{n}{train\PYZus{}data}\PY{p}{[}\PY{l+s+s1}{\PYZsq{}}\PY{l+s+s1}{WTISPLC}\PY{l+s+s1}{\PYZsq{}}\PY{p}{]}\PY{p}{;}
\end{Verbatim}


    We will now try to transform the time series which represents the output
emissions, with \(1\) representing an increase in the price from the
previous month and \(0\) representing a decrease in the price of the
oil.

    \begin{Verbatim}[commandchars=\\\{\}]
{\color{incolor}In [{\color{incolor}16}]:} \PY{n}{price\PYZus{}diff} \PY{o}{=} \PY{n}{price}\PY{o}{.}\PY{n}{diff}\PY{p}{(}\PY{p}{)}\PY{p}{[}\PY{l+m+mi}{1}\PY{p}{:}\PY{p}{]}\PY{p}{;} \PY{c+c1}{\PYZsh{} The first value is NaN as there is not a previous month to compare with}
         \PY{n}{e\PYZus{}seq} \PY{o}{=} \PY{n}{np}\PY{o}{.}\PY{n}{array}\PY{p}{(}\PY{n}{price\PYZus{}diff}\PY{o}{.}\PY{n}{apply}\PY{p}{(}\PY{k}{lambda} \PY{n}{x}\PY{p}{:} \PY{l+m+mi}{1} \PY{k}{if} \PY{n}{x} \PY{o}{\PYZgt{}} \PY{l+m+mi}{0} \PY{k}{else} \PY{l+m+mi}{0}\PY{p}{)}\PY{o}{.}\PY{n}{values}\PY{p}{)}\PY{p}{;} \PY{c+c1}{\PYZsh{} Replacing the change with 1 if positive, else 0}
\end{Verbatim}


    Given that we have obtained the output (observed emission sequence), we
can now use the \textbf{Baum-Welch algorithm} to learn the parameters of
the HMM generating this data.

We have earlier described the \textbf{Baum-Welch algorithm} in the
Background and Literature review, and we would be using the
implementation provided \href{https://github.com/lopatovsky/HMMs}{hmms}
to learn the parameters.

\textbf{IT IS VERY IMPORTANT} to note we can \textbf{only} use the
training data to train the HMM as we are assuming to be blind to the
testing data. However, we would be observing predictions on the
validation dataset and will tune our model to fit it, and we would be
using the testing dataframe to test the final performance of the tuned
model after validation.

We will create a model with random parameters, that will be eventually
trained to match the data - a discrete time HMM of three hidden states
(bull, bear, or stagnant) and two output variables (increase or
decrease).

    \begin{Verbatim}[commandchars=\\\{\}]
{\color{incolor}In [{\color{incolor}17}]:} \PY{n}{dhmm\PYZus{}r} \PY{o}{=} \PY{n}{hmms}\PY{o}{.}\PY{n}{DtHMM}\PY{o}{.}\PY{n}{random}\PY{p}{(}\PY{l+m+mi}{3} \PY{p}{,} \PY{l+m+mi}{2}\PY{p}{)}\PY{p}{;} 
\end{Verbatim}


    Given that the \(\texttt{hmms.DtHMM}\) takes a list of arrays no creater
than length \(32\), we will have to split our array in arrays each of
length \(32\) or less.

    \begin{Verbatim}[commandchars=\\\{\}]
{\color{incolor}In [{\color{incolor}18}]:} \PY{n}{e\PYZus{}seq} \PY{o}{=} \PY{n}{np}\PY{o}{.}\PY{n}{array\PYZus{}split}\PY{p}{(}\PY{n}{e\PYZus{}seq}\PY{p}{,} \PY{l+m+mi}{32}\PY{p}{)}\PY{p}{;}
\end{Verbatim}


    \hypertarget{baum-welch-algorithm}{%
\subsubsection{Baum-Welch Algorithm}\label{baum-welch-algorithm}}

We would now be using the \textbf{Baum-Welch algorithm} to learn the
parameters of the HMM generating the time-series.

The probability of the reestimated model after each iteration should
ideally be closer that the (unknown) generator's model, however chances
might be the estimation fell in the local optima.

Unfortunately, the financial time-series data do not have fixed
parameters, so the HMM has to be trained everytime when live-trading,
when using the \textbf{k-fold cross-validation} training method.

    \begin{Verbatim}[commandchars=\\\{\}]
{\color{incolor}In [{\color{incolor}19}]:} \PY{n}{dhmm\PYZus{}r}\PY{o}{.}\PY{n}{baum\PYZus{}welch}\PY{p}{(}\PY{n}{e\PYZus{}seq}\PY{p}{,} \PY{l+m+mi}{100}\PY{p}{)}\PY{p}{;} \PY{c+c1}{\PYZsh{} 100 iterations}
\end{Verbatim}

    We have now learnt the parameters generating the emission sequence.

    \begin{Verbatim}[commandchars=\\\{\}]
{\color{incolor}In [{\color{incolor}20}]:} \PY{n}{hmms}\PY{o}{.}\PY{n}{print\PYZus{}parameters}\PY{p}{(} \PY{n}{dhmm\PYZus{}r} \PY{p}{)}\PY{p}{;}
\end{Verbatim}


    \begin{Verbatim}[commandchars=\\\{\}]
Initial probabilities (π) :

    \end{Verbatim}

    
    \begin{verbatim}
          0
0  0.151149
1  0.203749
2  0.645102
    \end{verbatim}

    
    \begin{Verbatim}[commandchars=\\\{\}]
Transition probabilities matrix (A):

    \end{Verbatim}

    
    \begin{verbatim}
          0         1         2
0  0.556352  0.443599  0.000050
1  0.043233  0.382674  0.574092
2  0.550543  0.029073  0.420384
    \end{verbatim}

    
    \begin{Verbatim}[commandchars=\\\{\}]
Emission probabilities matrix (B):

    \end{Verbatim}

    
    \begin{verbatim}
          0         1
0  0.311665  0.688335
1  0.997841  0.002159
2  0.000163  0.999837
    \end{verbatim}

    
    \hypertarget{viterbi-algorithm}{%
\subsubsection{Viterbi Algorithm}\label{viterbi-algorithm}}

Now, given we now have parameters \(\lambda\) and the emitted
observation sequence \(\texttt{e\_seq}\), we can use the \textbf{Viterbi
Algorithm} to identify the most likely \textbf{state-transition} path
(i.e. \textbf{market regimes}) in the financial time-series.

    \begin{Verbatim}[commandchars=\\\{\}]
{\color{incolor}In [{\color{incolor}21}]:} \PY{p}{(} \PY{n}{log\PYZus{}prob}\PY{p}{,} \PY{n}{s\PYZus{}seq} \PY{p}{)} \PY{o}{=}  \PY{n}{dhmm\PYZus{}r}\PY{o}{.}\PY{n}{viterbi}\PY{p}{(}\PY{n}{np}\PY{o}{.}\PY{n}{concatenate}\PY{p}{(}\PY{n}{e\PYZus{}seq}\PY{p}{)}\PY{o}{.}\PY{n}{ravel}\PY{p}{(}\PY{p}{)}\PY{p}{)}\PY{p}{;}
\end{Verbatim}


    \hypertarget{multicolored-time-series-plot}{%
\subsubsection{Multicolored time series
plot}\label{multicolored-time-series-plot}}

Now, we will be plotting this graph in a \textbf{multicolored}
time-series plot to observe how well the regimes have been identified.

First, we will have to make a dataframe which has the both the price and
the associated regime the time-series data is in.

    \begin{Verbatim}[commandchars=\\\{\}]
{\color{incolor}In [{\color{incolor}22}]:} \PY{n}{price\PYZus{}plot} \PY{o}{=} \PY{n}{pd}\PY{o}{.}\PY{n}{DataFrame}\PY{p}{(}\PY{n}{price}\PY{p}{[}\PY{l+m+mi}{1}\PY{p}{:}\PY{p}{]}\PY{p}{,} \PY{n}{index}\PY{o}{=}\PY{n}{price}\PY{p}{[}\PY{l+m+mi}{1}\PY{p}{:}\PY{p}{]}\PY{o}{.}\PY{n}{index}\PY{p}{)}\PY{p}{;} \PY{c+c1}{\PYZsh{} Add price}
         \PY{n}{price\PYZus{}plot}\PY{p}{[}\PY{l+s+s1}{\PYZsq{}}\PY{l+s+s1}{Regime}\PY{l+s+s1}{\PYZsq{}}\PY{p}{]} \PY{o}{=} \PY{n}{s\PYZus{}seq}\PY{p}{;} \PY{c+c1}{\PYZsh{} Add a column representing the regime }
         \PY{n}{price\PYZus{}plot}\PY{p}{[}\PY{l+s+s1}{\PYZsq{}}\PY{l+s+s1}{diff}\PY{l+s+s1}{\PYZsq{}}\PY{p}{]} \PY{o}{=} \PY{n}{price\PYZus{}diff}\PY{p}{;} \PY{c+c1}{\PYZsh{} Add a column representing the increase or decrease in price}
\end{Verbatim}


    We do not know, however, which state represents which regime. Given that
the bull regimes should have a high positive change in price, bear
regimes should have a high negative change, and stagnant regimes are
closer to zero, we can use these properties to tell which state
represents which regime.

    \begin{Verbatim}[commandchars=\\\{\}]
{\color{incolor}In [{\color{incolor}23}]:} \PY{n}{means} \PY{o}{=} \PY{n}{price\PYZus{}plot}\PY{o}{.}\PY{n}{groupby}\PY{p}{(}\PY{p}{[}\PY{l+s+s1}{\PYZsq{}}\PY{l+s+s1}{Regime}\PY{l+s+s1}{\PYZsq{}}\PY{p}{]}\PY{p}{)}\PY{p}{[}\PY{l+s+s1}{\PYZsq{}}\PY{l+s+s1}{diff}\PY{l+s+s1}{\PYZsq{}}\PY{p}{]}\PY{o}{.}\PY{n}{mean}\PY{p}{(}\PY{p}{)}\PY{p}{;} \PY{c+c1}{\PYZsh{} Get means of all assigned states}
         \PY{n}{lst\PYZus{}1} \PY{o}{=} \PY{n}{means}\PY{o}{.}\PY{n}{index}\PY{o}{.}\PY{n}{tolist}\PY{p}{(}\PY{p}{)}\PY{p}{;}
         \PY{n}{lst\PYZus{}2} \PY{o}{=} \PY{n}{means}\PY{o}{.}\PY{n}{sort\PYZus{}values}\PY{p}{(}\PY{p}{)}\PY{o}{.}\PY{n}{index}\PY{o}{.}\PY{n}{tolist}\PY{p}{(}\PY{p}{)}\PY{p}{;}
         
         \PY{n}{map\PYZus{}regimes} \PY{o}{=} \PY{n+nb}{dict}\PY{p}{(}\PY{n+nb}{zip}\PY{p}{(}\PY{n}{lst\PYZus{}2}\PY{p}{,} \PY{n}{lst\PYZus{}1}\PY{p}{)}\PY{p}{)}\PY{p}{;}
         
         \PY{n}{price\PYZus{}plot}\PY{p}{[}\PY{l+s+s1}{\PYZsq{}}\PY{l+s+s1}{Regime}\PY{l+s+s1}{\PYZsq{}}\PY{p}{]} \PY{o}{=} \PY{n}{price\PYZus{}plot}\PY{p}{[}\PY{l+s+s1}{\PYZsq{}}\PY{l+s+s1}{Regime}\PY{l+s+s1}{\PYZsq{}}\PY{p}{]}\PY{o}{.}\PY{n}{map}\PY{p}{(}\PY{n}{map\PYZus{}regimes}\PY{p}{)}\PY{p}{;}
\end{Verbatim}


    Plotting the data as \textbf{multi-colored} time series:

    \begin{Verbatim}[commandchars=\\\{\}]
{\color{incolor}In [{\color{incolor}24}]:} \PY{k+kn}{import} \PY{n+nn}{matplotlib}\PY{n+nn}{.}\PY{n+nn}{dates} \PY{k}{as} \PY{n+nn}{mdates}
         \PY{k+kn}{import} \PY{n+nn}{matplotlib}\PY{n+nn}{.}\PY{n+nn}{patches} \PY{k}{as} \PY{n+nn}{mpatches}
         \PY{k+kn}{from} \PY{n+nn}{matplotlib}\PY{n+nn}{.}\PY{n+nn}{collections} \PY{k}{import} \PY{n}{LineCollection}
         \PY{k+kn}{from} \PY{n+nn}{matplotlib}\PY{n+nn}{.}\PY{n+nn}{colors} \PY{k}{import} \PY{n}{Colormap}\PY{p}{,} \PY{n}{ListedColormap}\PY{p}{,} \PY{n}{BoundaryNorm}
         
         \PY{n}{fig}\PY{p}{,} \PY{n}{ax1} \PY{o}{=} \PY{n}{plt}\PY{o}{.}\PY{n}{subplots}\PY{p}{(}\PY{n}{figsize}\PY{o}{=}\PY{p}{(}\PY{l+m+mi}{20}\PY{p}{,}\PY{l+m+mi}{8}\PY{p}{)}\PY{p}{)}\PY{p}{;}
         \PY{n}{ax}\PY{o}{.}\PY{n}{plot}\PY{p}{(}\PY{n}{price\PYZus{}plot}\PY{p}{[}\PY{l+s+s1}{\PYZsq{}}\PY{l+s+s1}{WTISPLC}\PY{l+s+s1}{\PYZsq{}}\PY{p}{]}\PY{p}{)}\PY{p}{;}
         
         \PY{n}{cmap}   \PY{o}{=} \PY{n}{ListedColormap}\PY{p}{(}\PY{p}{[}\PY{l+s+s1}{\PYZsq{}}\PY{l+s+s1}{r}\PY{l+s+s1}{\PYZsq{}}\PY{p}{,}\PY{l+s+s1}{\PYZsq{}}\PY{l+s+s1}{b}\PY{l+s+s1}{\PYZsq{}}\PY{p}{,}\PY{l+s+s1}{\PYZsq{}}\PY{l+s+s1}{g}\PY{l+s+s1}{\PYZsq{}}\PY{p}{]}\PY{p}{,}\PY{l+s+s1}{\PYZsq{}}\PY{l+s+s1}{indexed}\PY{l+s+s1}{\PYZsq{}}\PY{p}{)}\PY{p}{;} \PY{c+c1}{\PYZsh{} Make 0 (Bear) \PYZhy{} red, 1 (Stagnant) \PYZhy{} blue, 2 (Bull) \PYZhy{} green }
         \PY{n}{norm}   \PY{o}{=} \PY{n}{BoundaryNorm}\PY{p}{(}\PY{n+nb}{range}\PY{p}{(}\PY{l+m+mi}{3} \PY{o}{+} \PY{l+m+mi}{1}\PY{p}{)}\PY{p}{,} \PY{n}{cmap}\PY{o}{.}\PY{n}{N}\PY{p}{)}\PY{p}{;}
         \PY{n}{inxval} \PY{o}{=} \PY{n}{mdates}\PY{o}{.}\PY{n}{date2num}\PY{p}{(}\PY{n}{price\PYZus{}plot}\PY{p}{[}\PY{l+s+s1}{\PYZsq{}}\PY{l+s+s1}{WTISPLC}\PY{l+s+s1}{\PYZsq{}}\PY{p}{]}\PY{o}{.}\PY{n}{index}\PY{o}{.}\PY{n}{to\PYZus{}pydatetime}\PY{p}{(}\PY{p}{)}\PY{p}{)}\PY{p}{;}
         \PY{n}{points} \PY{o}{=} \PY{n}{np}\PY{o}{.}\PY{n}{array}\PY{p}{(}\PY{p}{[}\PY{n}{inxval}\PY{p}{,} \PY{n}{price\PYZus{}plot}\PY{p}{[}\PY{l+s+s1}{\PYZsq{}}\PY{l+s+s1}{WTISPLC}\PY{l+s+s1}{\PYZsq{}}\PY{p}{]}\PY{p}{]}\PY{p}{)}\PY{o}{.}\PY{n}{T}\PY{o}{.}\PY{n}{reshape}\PY{p}{(}\PY{o}{\PYZhy{}}\PY{l+m+mi}{1}\PY{p}{,} \PY{l+m+mi}{1}\PY{p}{,} \PY{l+m+mi}{2}\PY{p}{)}\PY{p}{;}
         \PY{n}{segments} \PY{o}{=} \PY{n}{np}\PY{o}{.}\PY{n}{concatenate}\PY{p}{(}\PY{p}{[}\PY{n}{points}\PY{p}{[}\PY{p}{:}\PY{o}{\PYZhy{}}\PY{l+m+mi}{1}\PY{p}{]}\PY{p}{,} \PY{n}{points}\PY{p}{[}\PY{l+m+mi}{1}\PY{p}{:}\PY{p}{]}\PY{p}{]}\PY{p}{,} \PY{n}{axis}\PY{o}{=}\PY{l+m+mi}{1}\PY{p}{)}\PY{p}{;}
         
         \PY{n}{lc} \PY{o}{=} \PY{n}{LineCollection}\PY{p}{(}\PY{n}{segments}\PY{p}{,} \PY{n}{cmap}\PY{o}{=}\PY{n}{cmap}\PY{p}{,} \PY{n}{norm}\PY{o}{=}\PY{n}{norm}\PY{p}{)}\PY{p}{;}
         \PY{n}{lc}\PY{o}{.}\PY{n}{set\PYZus{}array}\PY{p}{(}\PY{n}{price\PYZus{}plot}\PY{p}{[}\PY{l+s+s1}{\PYZsq{}}\PY{l+s+s1}{Regime}\PY{l+s+s1}{\PYZsq{}}\PY{p}{]}\PY{p}{)}\PY{p}{;}
         \PY{n}{plt}\PY{o}{.}\PY{n}{gca}\PY{p}{(}\PY{p}{)}\PY{o}{.}\PY{n}{add\PYZus{}collection}\PY{p}{(}\PY{n}{lc}\PY{p}{)}\PY{p}{;}
         \PY{n}{plt}\PY{o}{.}\PY{n}{xlim}\PY{p}{(}\PY{n}{price\PYZus{}plot}\PY{p}{[}\PY{l+s+s1}{\PYZsq{}}\PY{l+s+s1}{WTISPLC}\PY{l+s+s1}{\PYZsq{}}\PY{p}{]}\PY{o}{.}\PY{n}{index}\PY{o}{.}\PY{n}{min}\PY{p}{(}\PY{p}{)}\PY{p}{,} \PY{n}{price\PYZus{}plot}\PY{p}{[}\PY{l+s+s1}{\PYZsq{}}\PY{l+s+s1}{WTISPLC}\PY{l+s+s1}{\PYZsq{}}\PY{p}{]}\PY{o}{.}\PY{n}{index}\PY{o}{.}\PY{n}{max}\PY{p}{(}\PY{p}{)}\PY{p}{)}\PY{p}{;}
         \PY{n}{plt}\PY{o}{.}\PY{n}{ylim}\PY{p}{(}\PY{n}{price\PYZus{}plot}\PY{p}{[}\PY{l+s+s1}{\PYZsq{}}\PY{l+s+s1}{WTISPLC}\PY{l+s+s1}{\PYZsq{}}\PY{p}{]}\PY{o}{.}\PY{n}{min}\PY{p}{(}\PY{p}{)}\PY{p}{,} \PY{n}{price\PYZus{}plot}\PY{p}{[}\PY{l+s+s1}{\PYZsq{}}\PY{l+s+s1}{WTISPLC}\PY{l+s+s1}{\PYZsq{}}\PY{p}{]}\PY{o}{.}\PY{n}{max}\PY{p}{(}\PY{p}{)}\PY{p}{)}\PY{p}{;}
         
         \PY{n}{r\PYZus{}patch} \PY{o}{=} \PY{n}{mpatches}\PY{o}{.}\PY{n}{Patch}\PY{p}{(}\PY{n}{color}\PY{o}{=}\PY{l+s+s1}{\PYZsq{}}\PY{l+s+s1}{red}\PY{l+s+s1}{\PYZsq{}}\PY{p}{,} \PY{n}{label}\PY{o}{=}\PY{l+s+s1}{\PYZsq{}}\PY{l+s+s1}{Bear}\PY{l+s+s1}{\PYZsq{}}\PY{p}{)}\PY{p}{;}
         \PY{n}{g\PYZus{}patch} \PY{o}{=} \PY{n}{mpatches}\PY{o}{.}\PY{n}{Patch}\PY{p}{(}\PY{n}{color}\PY{o}{=}\PY{l+s+s1}{\PYZsq{}}\PY{l+s+s1}{green}\PY{l+s+s1}{\PYZsq{}}\PY{p}{,} \PY{n}{label}\PY{o}{=}\PY{l+s+s1}{\PYZsq{}}\PY{l+s+s1}{Bull}\PY{l+s+s1}{\PYZsq{}}\PY{p}{)}\PY{p}{;}
         \PY{n}{b\PYZus{}patch} \PY{o}{=} \PY{n}{mpatches}\PY{o}{.}\PY{n}{Patch}\PY{p}{(}\PY{n}{color}\PY{o}{=}\PY{l+s+s1}{\PYZsq{}}\PY{l+s+s1}{blue}\PY{l+s+s1}{\PYZsq{}}\PY{p}{,} \PY{n}{label}\PY{o}{=}\PY{l+s+s1}{\PYZsq{}}\PY{l+s+s1}{Stagnant}\PY{l+s+s1}{\PYZsq{}}\PY{p}{)}\PY{p}{;}
         
         \PY{n}{plt}\PY{o}{.}\PY{n}{legend}\PY{p}{(}\PY{n}{handles}\PY{o}{=}\PY{p}{[}\PY{n}{r\PYZus{}patch}\PY{p}{,} \PY{n}{g\PYZus{}patch}\PY{p}{,} \PY{n}{b\PYZus{}patch}\PY{p}{]}\PY{p}{)}\PY{p}{;}
         
         \PY{n}{plt}\PY{o}{.}\PY{n}{show}\PY{p}{(}\PY{p}{)}\PY{p}{;}
\end{Verbatim}


    \begin{center}
    \adjustimage{max size={0.9\linewidth}{0.9\paperheight}}{Implementation_files/Implementation_49_0.png}
    \end{center}
    { \hspace*{\fill} \\}
    
    We can observe from the graph that there are periods of bear runs, bull
runs, and stagnant periods. Now, we need to apply a similar method to
all the training data so that we can discretise our data and use it as
an input to our belief network.

\hypertarget{discretising-dataframes-using-hidden-markov-models}{%
\subsubsection{Discretising dataframes using Hidden Markov
Models}\label{discretising-dataframes-using-hidden-markov-models}}

We will write a function that trains an HMM, identifies the sequence of
hidden states, and then constructs a dataframe of all the variables and
stores it as training dataset.

    Now, we shall apply the same method to the entire training dataset so
that we can discretise it.

First, we shall learn all the parameters of Hidden Markov Models of all
the variables and store them.

    \begin{Verbatim}[commandchars=\\\{\}]
{\color{incolor}In [{\color{incolor}25}]:} \PY{k}{for} \PY{n}{series\PYZus{}id} \PY{o+ow}{in} \PY{n}{datasets}\PY{p}{:}
             \PY{k}{if} \PY{n}{series\PYZus{}id} \PY{o}{==} \PY{l+s+s1}{\PYZsq{}}\PY{l+s+s1}{forecast}\PY{l+s+s1}{\PYZsq{}}\PY{p}{:}
                 \PY{k}{break}\PY{p}{;}
             \PY{k}{else}\PY{p}{:}
                 \PY{n}{dhmm} \PY{o}{=} \PY{n}{hmms}\PY{o}{.}\PY{n}{DtHMM}\PY{o}{.}\PY{n}{random}\PY{p}{(}\PY{l+m+mi}{3}\PY{p}{,}\PY{l+m+mi}{2}\PY{p}{)}\PY{p}{;}
                 \PY{n}{data\PYZus{}diff} \PY{o}{=}  \PY{n}{train\PYZus{}data}\PY{p}{[}\PY{n}{series\PYZus{}id}\PY{p}{]}\PY{o}{.}\PY{n}{diff}\PY{p}{(}\PY{p}{)}\PY{p}{[}\PY{l+m+mi}{1}\PY{p}{:}\PY{p}{]}\PY{p}{;}
                 \PY{n}{emit\PYZus{}seq} \PY{o}{=} \PY{n}{np}\PY{o}{.}\PY{n}{array\PYZus{}split}\PY{p}{(}\PY{n}{data\PYZus{}diff}\PY{o}{.}\PY{n}{apply}\PY{p}{(}\PY{k}{lambda} \PY{n}{x}\PY{p}{:} \PY{l+m+mi}{1} \PY{k}{if} \PY{n}{x} \PY{o}{\PYZgt{}} \PY{l+m+mi}{0} \PY{k}{else} \PY{l+m+mi}{0}\PY{p}{)}\PY{o}{.}\PY{n}{values}\PY{p}{,} \PY{l+m+mi}{32}\PY{p}{)}\PY{p}{;}
                 \PY{n}{dhmm}\PY{o}{.}\PY{n}{baum\PYZus{}welch}\PY{p}{(}\PY{n}{emit\PYZus{}seq}\PY{p}{,} \PY{l+m+mi}{100}\PY{p}{)}\PY{p}{;}
                 \PY{n}{path} \PY{o}{=} \PY{l+s+s2}{\PYZdq{}}\PY{l+s+s2}{./hmms/}\PY{l+s+s2}{\PYZdq{}} \PY{o}{+} \PY{n}{series\PYZus{}id}\PY{o}{.}\PY{n}{replace}\PY{p}{(}\PY{l+s+s2}{\PYZdq{}}\PY{l+s+s2}{.}\PY{l+s+s2}{\PYZdq{}}\PY{p}{,} \PY{l+s+s2}{\PYZdq{}}\PY{l+s+s2}{\PYZus{}}\PY{l+s+s2}{\PYZdq{}}\PY{p}{)}\PY{p}{;}
                 \PY{n}{dhmm}\PY{o}{.}\PY{n}{save\PYZus{}params}\PY{p}{(}\PY{n}{path}\PY{p}{)}\PY{p}{;}
         
         \PY{c+c1}{\PYZsh{} To retrain a specific dataset you were not satisfied with}
         
         \PY{c+c1}{\PYZsh{} retrain\PYZus{}dataset = \PYZsq{}STEO.RGDPQ\PYZus{}NONOECD.M\PYZsq{};}
         \PY{c+c1}{\PYZsh{} dhmm = hmms.DtHMM.random(3,2);}
         \PY{c+c1}{\PYZsh{} data\PYZus{}diff =  train\PYZus{}data[retrain\PYZus{}dataset].diff()[1:];}
         \PY{c+c1}{\PYZsh{} emit\PYZus{}seq = np.array\PYZus{}split(data\PYZus{}diff.apply(lambda x: 1 if x \PYZgt{} 0 else 0).values, 32);}
         \PY{c+c1}{\PYZsh{} dhmm.baum\PYZus{}welch(emit\PYZus{}seq, 100);}
         \PY{c+c1}{\PYZsh{} path = \PYZdq{}./hmms/\PYZdq{} + retrain\PYZus{}dataset.replace(\PYZdq{}.\PYZdq{}, \PYZdq{}\PYZus{}\PYZdq{});}
         \PY{c+c1}{\PYZsh{} dhmm.save\PYZus{}params(path);}
         
         \PY{c+c1}{\PYZsh{} Comment out the code above if you have already learnt HMMs and do not want to relearn them}
\end{Verbatim}

    Now, we shall be constructing the discretised training dataframe.

    \begin{Verbatim}[commandchars=\\\{\}]
{\color{incolor}In [{\color{incolor}26}]:} \PY{n}{disc\PYZus{}test} \PY{o}{=} \PY{n}{pd}\PY{o}{.}\PY{n}{DataFrame}\PY{p}{(}\PY{n}{index} \PY{o}{=} \PY{n}{train\PYZus{}data}\PY{p}{[}\PY{l+m+mi}{1}\PY{p}{:}\PY{p}{]}\PY{o}{.}\PY{n}{index}\PY{p}{)}\PY{p}{;}
         
         \PY{k}{for} \PY{n}{series\PYZus{}id} \PY{o+ow}{in} \PY{n}{datasets}\PY{p}{:}
             \PY{n}{path} \PY{o}{=} \PY{l+s+s2}{\PYZdq{}}\PY{l+s+s2}{./hmms/}\PY{l+s+s2}{\PYZdq{}} \PY{o}{+} \PY{n}{series\PYZus{}id}\PY{o}{.}\PY{n}{replace}\PY{p}{(}\PY{l+s+s2}{\PYZdq{}}\PY{l+s+s2}{.}\PY{l+s+s2}{\PYZdq{}}\PY{p}{,} \PY{l+s+s2}{\PYZdq{}}\PY{l+s+s2}{\PYZus{}}\PY{l+s+s2}{\PYZdq{}}\PY{p}{)} \PY{o}{+} \PY{l+s+s2}{\PYZdq{}}\PY{l+s+s2}{.npz}\PY{l+s+s2}{\PYZdq{}}\PY{p}{;}
             \PY{k}{if} \PY{n}{series\PYZus{}id} \PY{o}{==} \PY{l+s+s1}{\PYZsq{}}\PY{l+s+s1}{forecast}\PY{l+s+s1}{\PYZsq{}}\PY{p}{:}
                 \PY{n}{dhmm} \PY{o}{=} \PY{n}{hmms}\PY{o}{.}\PY{n}{DtHMM}\PY{o}{.}\PY{n}{from\PYZus{}file}\PY{p}{(}\PY{l+s+s1}{\PYZsq{}}\PY{l+s+s1}{./hmms/WTISPLC.npz}\PY{l+s+s1}{\PYZsq{}}\PY{p}{)}\PY{p}{;}
             \PY{k}{else}\PY{p}{:}
                 \PY{n}{dhmm} \PY{o}{=} \PY{n}{hmms}\PY{o}{.}\PY{n}{DtHMM}\PY{o}{.}\PY{n}{from\PYZus{}file}\PY{p}{(}\PY{n}{path}\PY{p}{)}\PY{p}{;}
             \PY{n}{data\PYZus{}diff} \PY{o}{=}  \PY{n}{train\PYZus{}data}\PY{p}{[}\PY{n}{series\PYZus{}id}\PY{p}{]}\PY{o}{.}\PY{n}{diff}\PY{p}{(}\PY{p}{)}\PY{p}{[}\PY{l+m+mi}{1}\PY{p}{:}\PY{p}{]}\PY{p}{;}
             \PY{n}{emit\PYZus{}seq} \PY{o}{=} \PY{n}{np}\PY{o}{.}\PY{n}{array}\PY{p}{(}\PY{n}{data\PYZus{}diff}\PY{o}{.}\PY{n}{apply}\PY{p}{(}\PY{k}{lambda} \PY{n}{x}\PY{p}{:} \PY{l+m+mi}{1} \PY{k}{if} \PY{n}{x} \PY{o}{\PYZgt{}} \PY{l+m+mi}{0} \PY{k}{else} \PY{l+m+mi}{0}\PY{p}{)}\PY{o}{.}\PY{n}{values}\PY{p}{)}\PY{p}{;}
             \PY{p}{(} \PY{n}{log\PYZus{}prob}\PY{p}{,} \PY{n}{s\PYZus{}seq} \PY{p}{)} \PY{o}{=}  \PY{n}{dhmm}\PY{o}{.}\PY{n}{viterbi}\PY{p}{(}\PY{n}{emit\PYZus{}seq}\PY{p}{)}\PY{p}{;}
             \PY{n}{disc\PYZus{}test}\PY{p}{[}\PY{n}{series\PYZus{}id}\PY{p}{]} \PY{o}{=} \PY{n}{s\PYZus{}seq}\PY{p}{;}
         
         \PY{n}{disc\PYZus{}test}\PY{o}{.}\PY{n}{to\PYZus{}csv}\PY{p}{(}\PY{l+s+s2}{\PYZdq{}}\PY{l+s+s2}{./data/train\PYZus{}data.csv}\PY{l+s+s2}{\PYZdq{}}\PY{p}{)}\PY{p}{;} \PY{c+c1}{\PYZsh{} Saving to CSV}
\end{Verbatim}


    \hypertarget{recommended-plotting-regime-switch-plots}{%
\subsubsection{(Recommended) Plotting Regime Switch
plots}\label{recommended-plotting-regime-switch-plots}}

In order to verify if the \textbf{HMM} has correctly identified regimes
in all datasets, we should plot the \textbf{regime-switching models} of
all datasets and observe if the \textbf{bull}, \textbf{bear}, and
\textbf{stagnant} states have been correctly identified. The graphical
representation allows us to understand how the regimes have been learnt.

Omission of this step can result in the \textbf{HMM} being incorrectly
trained, hence identifying incorrect regimes consequently affecting the
belief network's training and prediction process.

    \begin{Verbatim}[commandchars=\\\{\}]
{\color{incolor}In [{\color{incolor}27}]:} \PY{n}{states} \PY{o}{=} \PY{n}{pd}\PY{o}{.}\PY{n}{read\PYZus{}csv}\PY{p}{(}\PY{l+s+s2}{\PYZdq{}}\PY{l+s+s2}{./data/train\PYZus{}data.csv}\PY{l+s+s2}{\PYZdq{}}\PY{p}{,} \PY{n}{index\PYZus{}col}\PY{o}{=}\PY{l+m+mi}{0}\PY{p}{)}\PY{p}{;}
         
         \PY{k}{for} \PY{n}{series\PYZus{}id} \PY{o+ow}{in} \PY{n}{datasets}\PY{p}{:}
             
                 \PY{n}{df} \PY{o}{=} \PY{n}{pd}\PY{o}{.}\PY{n}{DataFrame}\PY{p}{(}\PY{n}{index}\PY{o}{=}\PY{n}{train\PYZus{}data}\PY{p}{[}\PY{l+m+mi}{1}\PY{p}{:}\PY{p}{]}\PY{o}{.}\PY{n}{index}\PY{p}{)}\PY{p}{;}
                 \PY{n}{df}\PY{p}{[}\PY{n}{series\PYZus{}id}\PY{p}{]} \PY{o}{=} \PY{n}{train\PYZus{}data}\PY{p}{[}\PY{n}{series\PYZus{}id}\PY{p}{]}\PY{p}{[}\PY{l+m+mi}{1}\PY{p}{:}\PY{p}{]}\PY{p}{;}
                 \PY{n}{df}\PY{p}{[}\PY{l+s+s1}{\PYZsq{}}\PY{l+s+s1}{Diff}\PY{l+s+s1}{\PYZsq{}}\PY{p}{]} \PY{o}{=} \PY{n}{train\PYZus{}data}\PY{p}{[}\PY{n}{series\PYZus{}id}\PY{p}{]}\PY{o}{.}\PY{n}{diff}\PY{p}{(}\PY{p}{)}\PY{p}{[}\PY{l+m+mi}{1}\PY{p}{:}\PY{p}{]}\PY{p}{;}
                 \PY{n}{df}\PY{p}{[}\PY{l+s+s1}{\PYZsq{}}\PY{l+s+s1}{Regime}\PY{l+s+s1}{\PYZsq{}}\PY{p}{]} \PY{o}{=} \PY{n}{states}\PY{p}{[}\PY{n}{series\PYZus{}id}\PY{p}{]}\PY{p}{;}
                 
                 \PY{n}{means} \PY{o}{=} \PY{n}{df}\PY{o}{.}\PY{n}{groupby}\PY{p}{(}\PY{p}{[}\PY{l+s+s1}{\PYZsq{}}\PY{l+s+s1}{Regime}\PY{l+s+s1}{\PYZsq{}}\PY{p}{]}\PY{p}{)}\PY{p}{[}\PY{l+s+s1}{\PYZsq{}}\PY{l+s+s1}{Diff}\PY{l+s+s1}{\PYZsq{}}\PY{p}{]}\PY{o}{.}\PY{n}{mean}\PY{p}{(}\PY{p}{)}\PY{p}{;} \PY{c+c1}{\PYZsh{} Get means of all assigned states}
                 \PY{n}{lst\PYZus{}1} \PY{o}{=} \PY{n}{means}\PY{o}{.}\PY{n}{index}\PY{o}{.}\PY{n}{tolist}\PY{p}{(}\PY{p}{)}\PY{p}{;}
                 \PY{n}{lst\PYZus{}2} \PY{o}{=} \PY{n}{means}\PY{o}{.}\PY{n}{sort\PYZus{}values}\PY{p}{(}\PY{p}{)}\PY{o}{.}\PY{n}{index}\PY{o}{.}\PY{n}{tolist}\PY{p}{(}\PY{p}{)}\PY{p}{;}
         
                 \PY{n}{map\PYZus{}regimes} \PY{o}{=} \PY{n+nb}{dict}\PY{p}{(}\PY{n+nb}{zip}\PY{p}{(}\PY{n}{lst\PYZus{}2}\PY{p}{,} \PY{n}{lst\PYZus{}1}\PY{p}{)}\PY{p}{)}\PY{p}{;}
                 \PY{n}{df}\PY{p}{[}\PY{l+s+s1}{\PYZsq{}}\PY{l+s+s1}{Regime}\PY{l+s+s1}{\PYZsq{}}\PY{p}{]} \PY{o}{=} \PY{n}{df}\PY{p}{[}\PY{l+s+s1}{\PYZsq{}}\PY{l+s+s1}{Regime}\PY{l+s+s1}{\PYZsq{}}\PY{p}{]}\PY{o}{.}\PY{n}{map}\PY{p}{(}\PY{n}{map\PYZus{}regimes}\PY{p}{)}\PY{p}{;}
                 
                 
                 \PY{n}{cmap}   \PY{o}{=} \PY{n}{ListedColormap}\PY{p}{(}\PY{p}{[}\PY{l+s+s1}{\PYZsq{}}\PY{l+s+s1}{r}\PY{l+s+s1}{\PYZsq{}}\PY{p}{,}\PY{l+s+s1}{\PYZsq{}}\PY{l+s+s1}{b}\PY{l+s+s1}{\PYZsq{}}\PY{p}{,}\PY{l+s+s1}{\PYZsq{}}\PY{l+s+s1}{g}\PY{l+s+s1}{\PYZsq{}}\PY{p}{]}\PY{p}{,}\PY{l+s+s1}{\PYZsq{}}\PY{l+s+s1}{indexed}\PY{l+s+s1}{\PYZsq{}}\PY{p}{)}\PY{p}{;}
                 \PY{n}{norm}   \PY{o}{=} \PY{n}{BoundaryNorm}\PY{p}{(}\PY{n+nb}{range}\PY{p}{(}\PY{l+m+mi}{3} \PY{o}{+} \PY{l+m+mi}{1}\PY{p}{)}\PY{p}{,} \PY{n}{cmap}\PY{o}{.}\PY{n}{N}\PY{p}{)}\PY{p}{;}
                 \PY{n}{inxval} \PY{o}{=} \PY{n}{mdates}\PY{o}{.}\PY{n}{date2num}\PY{p}{(}\PY{n}{df}\PY{p}{[}\PY{n}{series\PYZus{}id}\PY{p}{]}\PY{o}{.}\PY{n}{index}\PY{o}{.}\PY{n}{to\PYZus{}pydatetime}\PY{p}{(}\PY{p}{)}\PY{p}{)}\PY{p}{;}
                 \PY{n}{points} \PY{o}{=} \PY{n}{np}\PY{o}{.}\PY{n}{array}\PY{p}{(}\PY{p}{[}\PY{n}{inxval}\PY{p}{,} \PY{n}{df}\PY{p}{[}\PY{n}{series\PYZus{}id}\PY{p}{]}\PY{p}{]}\PY{p}{)}\PY{o}{.}\PY{n}{T}\PY{o}{.}\PY{n}{reshape}\PY{p}{(}\PY{o}{\PYZhy{}}\PY{l+m+mi}{1}\PY{p}{,} \PY{l+m+mi}{1}\PY{p}{,} \PY{l+m+mi}{2}\PY{p}{)}\PY{p}{;}
                 \PY{n}{segments} \PY{o}{=} \PY{n}{np}\PY{o}{.}\PY{n}{concatenate}\PY{p}{(}\PY{p}{[}\PY{n}{points}\PY{p}{[}\PY{p}{:}\PY{o}{\PYZhy{}}\PY{l+m+mi}{1}\PY{p}{]}\PY{p}{,} \PY{n}{points}\PY{p}{[}\PY{l+m+mi}{1}\PY{p}{:}\PY{p}{]}\PY{p}{]}\PY{p}{,} \PY{n}{axis}\PY{o}{=}\PY{l+m+mi}{1}\PY{p}{)}\PY{p}{;}
         
                 \PY{n}{lc} \PY{o}{=} \PY{n}{LineCollection}\PY{p}{(}\PY{n}{segments}\PY{p}{,} \PY{n}{cmap}\PY{o}{=}\PY{n}{cmap}\PY{p}{,} \PY{n}{norm}\PY{o}{=}\PY{n}{norm}\PY{p}{)}\PY{p}{;}
                 \PY{n}{lc}\PY{o}{.}\PY{n}{set\PYZus{}array}\PY{p}{(}\PY{n}{df}\PY{p}{[}\PY{l+s+s1}{\PYZsq{}}\PY{l+s+s1}{Regime}\PY{l+s+s1}{\PYZsq{}}\PY{p}{]}\PY{p}{)}\PY{p}{;}
                 \PY{n}{plt}\PY{o}{.}\PY{n}{gca}\PY{p}{(}\PY{p}{)}\PY{o}{.}\PY{n}{add\PYZus{}collection}\PY{p}{(}\PY{n}{lc}\PY{p}{)}\PY{p}{;}
                 \PY{n}{plt}\PY{o}{.}\PY{n}{xlim}\PY{p}{(}\PY{n}{df}\PY{p}{[}\PY{n}{series\PYZus{}id}\PY{p}{]}\PY{o}{.}\PY{n}{index}\PY{o}{.}\PY{n}{min}\PY{p}{(}\PY{p}{)}\PY{p}{,} \PY{n}{df}\PY{p}{[}\PY{n}{series\PYZus{}id}\PY{p}{]}\PY{o}{.}\PY{n}{index}\PY{o}{.}\PY{n}{max}\PY{p}{(}\PY{p}{)}\PY{p}{)}\PY{p}{;}
                 \PY{n}{plt}\PY{o}{.}\PY{n}{ylim}\PY{p}{(}\PY{n}{df}\PY{p}{[}\PY{n}{series\PYZus{}id}\PY{p}{]}\PY{o}{.}\PY{n}{min}\PY{p}{(}\PY{p}{)}\PY{p}{,} \PY{n}{df}\PY{p}{[}\PY{n}{series\PYZus{}id}\PY{p}{]}\PY{o}{.}\PY{n}{max}\PY{p}{(}\PY{p}{)}\PY{p}{)}\PY{p}{;}
         
                 \PY{n}{r\PYZus{}patch} \PY{o}{=} \PY{n}{mpatches}\PY{o}{.}\PY{n}{Patch}\PY{p}{(}\PY{n}{color}\PY{o}{=}\PY{l+s+s1}{\PYZsq{}}\PY{l+s+s1}{red}\PY{l+s+s1}{\PYZsq{}}\PY{p}{,} \PY{n}{label}\PY{o}{=}\PY{l+s+s1}{\PYZsq{}}\PY{l+s+s1}{Bear}\PY{l+s+s1}{\PYZsq{}}\PY{p}{)}\PY{p}{;}
                 \PY{n}{g\PYZus{}patch} \PY{o}{=} \PY{n}{mpatches}\PY{o}{.}\PY{n}{Patch}\PY{p}{(}\PY{n}{color}\PY{o}{=}\PY{l+s+s1}{\PYZsq{}}\PY{l+s+s1}{green}\PY{l+s+s1}{\PYZsq{}}\PY{p}{,} \PY{n}{label}\PY{o}{=}\PY{l+s+s1}{\PYZsq{}}\PY{l+s+s1}{Bull}\PY{l+s+s1}{\PYZsq{}}\PY{p}{)}\PY{p}{;}
                 \PY{n}{b\PYZus{}patch} \PY{o}{=} \PY{n}{mpatches}\PY{o}{.}\PY{n}{Patch}\PY{p}{(}\PY{n}{color}\PY{o}{=}\PY{l+s+s1}{\PYZsq{}}\PY{l+s+s1}{blue}\PY{l+s+s1}{\PYZsq{}}\PY{p}{,} \PY{n}{label}\PY{o}{=}\PY{l+s+s1}{\PYZsq{}}\PY{l+s+s1}{Stagnant}\PY{l+s+s1}{\PYZsq{}}\PY{p}{)}\PY{p}{;}
         
                 \PY{n}{plt}\PY{o}{.}\PY{n}{legend}\PY{p}{(}\PY{n}{handles}\PY{o}{=}\PY{p}{[}\PY{n}{r\PYZus{}patch}\PY{p}{,} \PY{n}{g\PYZus{}patch}\PY{p}{,} \PY{n}{b\PYZus{}patch}\PY{p}{]}\PY{p}{)}\PY{p}{;}
         
                 \PY{n}{name} \PY{o}{=} \PY{l+s+s2}{\PYZdq{}}\PY{l+s+s2}{./plots/}\PY{l+s+s2}{\PYZdq{}} \PY{o}{+} \PY{n}{series\PYZus{}id}\PY{o}{.}\PY{n}{replace}\PY{p}{(}\PY{l+s+s2}{\PYZdq{}}\PY{l+s+s2}{.}\PY{l+s+s2}{\PYZdq{}}\PY{p}{,} \PY{l+s+s2}{\PYZdq{}}\PY{l+s+s2}{\PYZus{}}\PY{l+s+s2}{\PYZdq{}}\PY{p}{)} \PY{o}{+} \PY{l+s+s2}{\PYZdq{}}\PY{l+s+s2}{.png}\PY{l+s+s2}{\PYZdq{}}\PY{p}{;}
         
                 \PY{n}{plt}\PY{o}{.}\PY{n}{savefig}\PY{p}{(}\PY{n}{name}\PY{p}{)}\PY{p}{;}
                 \PY{n}{plt}\PY{o}{.}\PY{n}{close}\PY{p}{(}\PY{p}{)}\PY{p}{;}
\end{Verbatim}


    We have sucessfully discretised our training dataset and would now be
using it to train the Belief Network.

    \hypertarget{learning-bayesian-network-using-hill-climbing}{%
\subsubsection{Learning Bayesian Network using Hill
Climbing}\label{learning-bayesian-network-using-hill-climbing}}

Given that we have discretised our training dataset, we can now use
\href{http://pgmpy.org/}{pgmpy} to construct a belief network of all the
variables.

We would be using the \textbf{Hill Climbing} approach to learn the
belief network. We have given a brief description of the Hill Climbing
algorithm in the thesis and would now be using its implementation pgmpy
to learn the structure of the oil markets.

We shall begin with importing the relevant modules from pgmpy in Python.

Given we would be using the \emph{BIC Scoring Algorithm} as the scoring
function for Hill Climbing, we will import the relevant functions from
the relevant module in pgmpy.

    \begin{Verbatim}[commandchars=\\\{\}]
{\color{incolor}In [{\color{incolor}28}]:} \PY{k+kn}{from} \PY{n+nn}{pgmpy}\PY{n+nn}{.}\PY{n+nn}{models} \PY{k}{import} \PY{n}{BayesianModel}
         \PY{k+kn}{from} \PY{n+nn}{pgmpy}\PY{n+nn}{.}\PY{n+nn}{estimators} \PY{k}{import} \PY{n}{HillClimbSearch}
         \PY{k+kn}{from} \PY{n+nn}{pgmpy}\PY{n+nn}{.}\PY{n+nn}{estimators} \PY{k}{import} \PY{n}{BayesianEstimator}
         \PY{k+kn}{from} \PY{n+nn}{pgmpy}\PY{n+nn}{.}\PY{n+nn}{estimators} \PY{k}{import} \PY{n}{BicScore}\PY{p}{,} \PY{n}{K2Score}\PY{p}{,} \PY{n}{BdeuScore}
         
         \PY{n}{train\PYZus{}data} \PY{o}{=} \PY{n}{pd}\PY{o}{.}\PY{n}{read\PYZus{}csv}\PY{p}{(}\PY{l+s+s2}{\PYZdq{}}\PY{l+s+s2}{./data/train\PYZus{}data.csv}\PY{l+s+s2}{\PYZdq{}}\PY{p}{,} \PY{n}{index\PYZus{}col}\PY{o}{=}\PY{l+m+mi}{0}\PY{p}{)}\PY{p}{;} \PY{c+c1}{\PYZsh{} Retrieve training set}
\end{Verbatim}


    We shall now be performing a Hill-Climbing search. As difficult as it
seems, it is a quite straightforward process.

We construct a instance of a Bayesian model, having the initial
structure constructed by expert knowledge from EIA. We shall then use
Hill Climbing to `attach' remaining macroeconomic variables to the
constructed model.

    \begin{Verbatim}[commandchars=\\\{\}]
{\color{incolor}In [{\color{incolor}29}]:} \PY{n}{hc} \PY{o}{=} \PY{n}{HillClimbSearch}\PY{p}{(}\PY{n}{train\PYZus{}data}\PY{p}{,} \PY{n}{scoring\PYZus{}method}\PY{o}{=}\PY{n}{K2Score}\PY{p}{(}\PY{n}{train\PYZus{}data}\PY{p}{)}\PY{p}{)}\PY{p}{;} \PY{c+c1}{\PYZsh{}  Initialise Hill Climbing Estimator}
         \PY{n}{expert} \PY{o}{=}  \PY{n}{BayesianModel}\PY{p}{(}\PY{p}{)}\PY{p}{;}
         \PY{n}{expert}\PY{o}{.}\PY{n}{add\PYZus{}nodes\PYZus{}from}\PY{p}{(}\PY{n}{datasets}\PY{p}{)}\PY{p}{;}
         \PY{n}{expert}\PY{o}{.}\PY{n}{add\PYZus{}edges\PYZus{}from}\PY{p}{(}\PY{p}{[}
                                         \PY{p}{(}\PY{l+s+s1}{\PYZsq{}}\PY{l+s+s1}{STEO.PAPR\PYZus{}NONOPEC.M}\PY{l+s+s1}{\PYZsq{}}\PY{p}{,} \PY{l+s+s1}{\PYZsq{}}\PY{l+s+s1}{WTISPLC}\PY{l+s+s1}{\PYZsq{}}\PY{p}{)}\PY{p}{,} 
                                         \PY{p}{(}\PY{l+s+s1}{\PYZsq{}}\PY{l+s+s1}{STEO.PAPR\PYZus{}OPEC.M}\PY{l+s+s1}{\PYZsq{}}\PY{p}{,} \PY{l+s+s1}{\PYZsq{}}\PY{l+s+s1}{WTISPLC}\PY{l+s+s1}{\PYZsq{}}\PY{p}{)}\PY{p}{,}
                                         \PY{p}{(}\PY{l+s+s1}{\PYZsq{}}\PY{l+s+s1}{STEO.PATC\PYZus{}OECD.M}\PY{l+s+s1}{\PYZsq{}}\PY{p}{,} \PY{l+s+s1}{\PYZsq{}}\PY{l+s+s1}{WTISPLC}\PY{l+s+s1}{\PYZsq{}}\PY{p}{)}\PY{p}{,}
                                         \PY{p}{(}\PY{l+s+s1}{\PYZsq{}}\PY{l+s+s1}{STEO.PATC\PYZus{}NON\PYZus{}OECD.M}\PY{l+s+s1}{\PYZsq{}}\PY{p}{,} \PY{l+s+s1}{\PYZsq{}}\PY{l+s+s1}{WTISPLC}\PY{l+s+s1}{\PYZsq{}}\PY{p}{)}\PY{p}{,}
                                         \PY{p}{(}\PY{l+s+s1}{\PYZsq{}}\PY{l+s+s1}{STEO.RGDPQ\PYZus{}OECD.M}\PY{l+s+s1}{\PYZsq{}}\PY{p}{,} \PY{l+s+s1}{\PYZsq{}}\PY{l+s+s1}{STEO.PATC\PYZus{}OECD.M}\PY{l+s+s1}{\PYZsq{}}\PY{p}{)}\PY{p}{,}
                                         \PY{p}{(}\PY{l+s+s1}{\PYZsq{}}\PY{l+s+s1}{STEO.RGDPQ\PYZus{}NONOECD.M}\PY{l+s+s1}{\PYZsq{}}\PY{p}{,} \PY{l+s+s1}{\PYZsq{}}\PY{l+s+s1}{STEO.PATC\PYZus{}NON\PYZus{}OECD.M}\PY{l+s+s1}{\PYZsq{}}\PY{p}{)}\PY{p}{,}
                               \PY{p}{]}\PY{p}{)}\PY{p}{;} 
         
         \PY{n}{model} \PY{o}{=} \PY{n}{hc}\PY{o}{.}\PY{n}{estimate}\PY{p}{(}\PY{n}{expert}\PY{p}{)}\PY{p}{;} \PY{c+c1}{\PYZsh{} Performs local hill climb search}
         
         \PY{n}{model}\PY{o}{.}\PY{n}{fit}\PY{p}{(}\PY{n}{train\PYZus{}data}\PY{p}{,}
                   \PY{n}{state\PYZus{}names}\PY{o}{=}\PY{n+nb}{dict}\PY{p}{(}\PY{n+nb}{map}\PY{p}{(}\PY{k}{lambda} \PY{n}{e}\PY{p}{:} \PY{p}{(}\PY{n}{e}\PY{p}{,} \PY{p}{[}\PY{l+m+mi}{0}\PY{p}{,} \PY{l+m+mi}{1}\PY{p}{,} \PY{l+m+mi}{2}\PY{p}{]}\PY{p}{)}\PY{p}{,} \PY{n}{datasets}\PY{p}{)}\PY{p}{)}\PY{p}{,} 
                   \PY{n}{estimator}\PY{o}{=}\PY{n}{BayesianEstimator}\PY{p}{,} \PY{n}{prior\PYZus{}type}\PY{o}{=}\PY{l+s+s2}{\PYZdq{}}\PY{l+s+s2}{K2}\PY{l+s+s2}{\PYZdq{}}\PY{p}{)}\PY{p}{;}
\end{Verbatim}


    \begin{Verbatim}[commandchars=\\\{\}]
{\color{incolor}In [{\color{incolor}46}]:} \PY{n+nb}{print}\PY{p}{(}\PY{n}{model}\PY{o}{.}\PY{n}{edges}\PY{p}{(}\PY{p}{)}\PY{p}{)}\PY{p}{;}
         
         \PY{k+kn}{import} \PY{n+nn}{networkx} \PY{k}{as} \PY{n+nn}{nx}
         \PY{k+kn}{import} \PY{n+nn}{pylab} \PY{k}{as} \PY{n+nn}{plt}
         
         
         \PY{n}{G}\PY{o}{=}\PY{n}{nx}\PY{o}{.}\PY{n}{Graph}\PY{p}{(}\PY{p}{)}\PY{p}{;}
         \PY{n}{G}\PY{o}{.}\PY{n}{add\PYZus{}edges\PYZus{}from}\PY{p}{(}\PY{n}{model}\PY{o}{.}\PY{n}{edges}\PY{p}{(}\PY{p}{)}\PY{p}{)}\PY{p}{;}
         \PY{n}{pos} \PY{o}{=} \PY{n}{nx}\PY{o}{.}\PY{n}{spring\PYZus{}layout}\PY{p}{(}\PY{n}{G}\PY{p}{)}\PY{p}{;}
         \PY{n}{nx}\PY{o}{.}\PY{n}{draw\PYZus{}networkx\PYZus{}nodes}\PY{p}{(}\PY{n}{G}\PY{p}{,} \PY{n}{pos}\PY{p}{,} \PY{n}{node\PYZus{}size} \PY{o}{=} \PY{l+m+mi}{10}\PY{p}{)}\PY{p}{;}
         \PY{n}{nx}\PY{o}{.}\PY{n}{draw\PYZus{}networkx\PYZus{}edges}\PY{p}{(}\PY{n}{G}\PY{p}{,} \PY{n}{pos}\PY{p}{,} \PY{n}{arrows}\PY{o}{=}\PY{k+kc}{True}\PY{p}{)}\PY{p}{;}
         \PY{n}{nx}\PY{o}{.}\PY{n}{draw\PYZus{}networkx\PYZus{}labels}\PY{p}{(}\PY{n}{G}\PY{p}{,} \PY{n}{pos}\PY{p}{)}\PY{p}{;}
         \PY{n}{plt}\PY{o}{.}\PY{n}{figure}\PY{p}{(}\PY{l+m+mi}{5}\PY{p}{,}\PY{n}{figsize}\PY{o}{=}\PY{p}{(}\PY{l+m+mi}{20}\PY{p}{,}\PY{l+m+mi}{10}\PY{p}{)}\PY{p}{)}\PY{p}{;}
\end{Verbatim}


    \begin{Verbatim}[commandchars=\\\{\}]
[('STEO.RGDPQ\_NONOECD.M', 'STEO.PATC\_NON\_OECD.M'), ('STEO.RGDPQ\_OECD.M', 'STEO.COPC\_OPEC.M'), ('STEO.RGDPQ\_OECD.M', 'STEO.CORIPUS.M'), ('STEO.RGDPQ\_OECD.M', 'STEO.FOREX\_WORLD.M'), ('STEO.RGDPQ\_OECD.M', 'IPN213111N'), ('STEO.RGDPQ\_OECD.M', 'STEO.COPRPUS.M'), ('STEO.RGDPQ\_OECD.M', 'WTISPLC'), ('STEO.RGDPQ\_OECD.M', 'forecast'), ('STEO.RGDPQ\_OECD.M', 'STEO.T3\_STCHANGE\_NOECD.M'), ('STEO.PAPR\_NONOPEC.M', 'WTISPLC'), ('STEO.PAPR\_OPEC.M', 'STEO.T3\_STCHANGE\_NOECD.M'), ('STEO.PAPR\_OPEC.M', 'PET.MCRIMXX2.M'), ('STEO.PATC\_OECD.M', 'WTISPLC'), ('STEO.PATC\_OECD.M', 'CAPG211S'), ('STEO.COPRPUS.M', 'IPG211111CN'), ('STEO.COPRPUS.M', 'CAPUTLG211S'), ('STEO.COPRPUS.M', 'STEO.PAPR\_NONOPEC.M'), ('STEO.FOREX\_WORLD.M', 'STEO.COPC\_OPEC.M'), ('STEO.PASC\_OECD\_T3.M', 'STEO.T3\_STCHANGE\_OOECD.M'), ('STEO.PASC\_OECD\_T3.M', 'STEO.PATC\_OECD.M'), ('STEO.COPC\_OPEC.M', 'STEO.PAPR\_OPEC.M'), ('STEO.T3\_STCHANGE\_NOECD.M', 'STEO.PASC\_OECD\_T3.M'), ('CPIENGSL', 'PCU211211'), ('CAPG211S', 'STEO.CORIPUS.M'), ('CAPG211S', 'WTISPLC'), ('CAPUTLG211S', 'IPG211S'), ('CAPUTLG211S', 'STEO.PATC\_NON\_OECD.M'), ('IPG211S', 'STEO.T3\_STCHANGE\_OOECD.M'), ('IPG211111CN', 'WTISPLC'), ('IPG211111CN', 'STEO.PAPR\_NONOPEC.M'), ('IPN213111N', 'WTISPLC'), ('WTISPLC', 'forecast'), ('WTISPLC', 'CPIENGSL'), ('forecast', 'CPIENGSL')]

    \end{Verbatim}

    \begin{center}
    \adjustimage{max size={0.9\linewidth}{0.9\paperheight}}{Implementation_files/Implementation_62_1.png}
    \end{center}
    { \hspace*{\fill} \\}
    
    
    \begin{verbatim}
<matplotlib.figure.Figure at 0x11a594588>
    \end{verbatim}

    
    We can also represent our model in the form of a directed graph, using
NetworkX.

    We have now fitted our model using the Hill Climb search, and now can
make inferences using forecasts as evidence.

\hypertarget{validation}{%
\subsection{Validation}\label{validation}}

We have successfully fitted our model to the data and need to test our
model now.

We will now be using our \textbf{validation} dataset and would be making
predictions on that and would be accordingly adjusting our model if we
are not satisfied by the performance.

We first have to discretise the validation dataset. We would be using
the HMMs trained on the \textbf{training set}. We \textbf{CANNOT} use
the \textbf{validation set} to train anything, including HMMs!

    \begin{Verbatim}[commandchars=\\\{\}]
{\color{incolor}In [{\color{incolor}31}]:} \PY{n}{discrete\PYZus{}vald} \PY{o}{=} \PY{n}{pd}\PY{o}{.}\PY{n}{DataFrame}\PY{p}{(}\PY{n}{index} \PY{o}{=} \PY{n}{vald\PYZus{}data}\PY{p}{[}\PY{l+m+mi}{1}\PY{p}{:}\PY{p}{]}\PY{o}{.}\PY{n}{index}\PY{p}{)}\PY{p}{;}
         
         \PY{k}{for} \PY{n}{series\PYZus{}id} \PY{o+ow}{in} \PY{n}{datasets}\PY{p}{:}
             \PY{n}{path} \PY{o}{=} \PY{l+s+s2}{\PYZdq{}}\PY{l+s+s2}{./hmms/}\PY{l+s+s2}{\PYZdq{}} \PY{o}{+} \PY{n}{series\PYZus{}id}\PY{o}{.}\PY{n}{replace}\PY{p}{(}\PY{l+s+s2}{\PYZdq{}}\PY{l+s+s2}{.}\PY{l+s+s2}{\PYZdq{}}\PY{p}{,} \PY{l+s+s2}{\PYZdq{}}\PY{l+s+s2}{\PYZus{}}\PY{l+s+s2}{\PYZdq{}}\PY{p}{)} \PY{o}{+} \PY{l+s+s2}{\PYZdq{}}\PY{l+s+s2}{.npz}\PY{l+s+s2}{\PYZdq{}}\PY{p}{;}
             \PY{k}{if} \PY{n}{series\PYZus{}id} \PY{o}{==} \PY{l+s+s1}{\PYZsq{}}\PY{l+s+s1}{forecast}\PY{l+s+s1}{\PYZsq{}}\PY{p}{:}
                 \PY{n}{dhmm} \PY{o}{=} \PY{n}{hmms}\PY{o}{.}\PY{n}{DtHMM}\PY{o}{.}\PY{n}{from\PYZus{}file}\PY{p}{(}\PY{l+s+s1}{\PYZsq{}}\PY{l+s+s1}{./hmms/WTISPLC.npz}\PY{l+s+s1}{\PYZsq{}}\PY{p}{)}\PY{p}{;}
             \PY{k}{else}\PY{p}{:}
                 \PY{n}{dhmm} \PY{o}{=} \PY{n}{hmms}\PY{o}{.}\PY{n}{DtHMM}\PY{o}{.}\PY{n}{from\PYZus{}file}\PY{p}{(}\PY{n}{path}\PY{p}{)}\PY{p}{;}
             \PY{n}{data\PYZus{}diff} \PY{o}{=}  \PY{n}{vald\PYZus{}data}\PY{p}{[}\PY{n}{series\PYZus{}id}\PY{p}{]}\PY{o}{.}\PY{n}{diff}\PY{p}{(}\PY{p}{)}\PY{p}{[}\PY{l+m+mi}{1}\PY{p}{:}\PY{p}{]}\PY{p}{;}
             \PY{n}{emit\PYZus{}seq} \PY{o}{=} \PY{n}{np}\PY{o}{.}\PY{n}{array}\PY{p}{(}\PY{n}{data\PYZus{}diff}\PY{o}{.}\PY{n}{apply}\PY{p}{(}\PY{k}{lambda} \PY{n}{x}\PY{p}{:} \PY{l+m+mi}{1} \PY{k}{if} \PY{n}{x} \PY{o}{\PYZgt{}} \PY{l+m+mi}{0} \PY{k}{else} \PY{l+m+mi}{0}\PY{p}{)}\PY{o}{.}\PY{n}{values}\PY{p}{)}\PY{p}{;}
             \PY{p}{(} \PY{n}{log\PYZus{}prob}\PY{p}{,} \PY{n}{s\PYZus{}seq} \PY{p}{)} \PY{o}{=}  \PY{n}{dhmm}\PY{o}{.}\PY{n}{viterbi}\PY{p}{(}\PY{n}{emit\PYZus{}seq}\PY{p}{)}\PY{p}{;}
             \PY{n}{discrete\PYZus{}vald}\PY{p}{[}\PY{n}{series\PYZus{}id}\PY{p}{]} \PY{o}{=} \PY{n}{s\PYZus{}seq}\PY{p}{;}
         
         \PY{n}{discrete\PYZus{}vald}\PY{o}{.}\PY{n}{to\PYZus{}csv}\PY{p}{(}\PY{l+s+s2}{\PYZdq{}}\PY{l+s+s2}{./data/validation\PYZus{}data.csv}\PY{l+s+s2}{\PYZdq{}}\PY{p}{)}\PY{p}{;} \PY{c+c1}{\PYZsh{} Saving to CSV}
\end{Verbatim}


    Now, we shall be plotting this data to see how well the trained HMMs
predicted regimes on the valdiation dataset.

    \begin{Verbatim}[commandchars=\\\{\}]
{\color{incolor}In [{\color{incolor}32}]:} \PY{n}{states} \PY{o}{=} \PY{n}{pd}\PY{o}{.}\PY{n}{read\PYZus{}csv}\PY{p}{(}\PY{l+s+s2}{\PYZdq{}}\PY{l+s+s2}{./data/validation\PYZus{}data.csv}\PY{l+s+s2}{\PYZdq{}}\PY{p}{,} \PY{n}{index\PYZus{}col}\PY{o}{=}\PY{l+m+mi}{0}\PY{p}{)}\PY{p}{;}
         
         \PY{k}{for} \PY{n}{series\PYZus{}id} \PY{o+ow}{in} \PY{n}{datasets}\PY{p}{:}
             
                 \PY{n}{df} \PY{o}{=} \PY{n}{pd}\PY{o}{.}\PY{n}{DataFrame}\PY{p}{(}\PY{n}{index}\PY{o}{=}\PY{n}{vald\PYZus{}data}\PY{p}{[}\PY{l+m+mi}{1}\PY{p}{:}\PY{p}{]}\PY{o}{.}\PY{n}{index}\PY{p}{)}\PY{p}{;}
                 \PY{n}{df}\PY{p}{[}\PY{n}{series\PYZus{}id}\PY{p}{]} \PY{o}{=} \PY{n}{vald\PYZus{}data}\PY{p}{[}\PY{n}{series\PYZus{}id}\PY{p}{]}\PY{p}{[}\PY{l+m+mi}{1}\PY{p}{:}\PY{p}{]}\PY{p}{;}
                 \PY{n}{df}\PY{p}{[}\PY{l+s+s1}{\PYZsq{}}\PY{l+s+s1}{Diff}\PY{l+s+s1}{\PYZsq{}}\PY{p}{]} \PY{o}{=} \PY{n}{vald\PYZus{}data}\PY{p}{[}\PY{n}{series\PYZus{}id}\PY{p}{]}\PY{o}{.}\PY{n}{diff}\PY{p}{(}\PY{p}{)}\PY{p}{[}\PY{l+m+mi}{1}\PY{p}{:}\PY{p}{]}\PY{p}{;}
                 \PY{n}{df}\PY{p}{[}\PY{l+s+s1}{\PYZsq{}}\PY{l+s+s1}{Regime}\PY{l+s+s1}{\PYZsq{}}\PY{p}{]} \PY{o}{=} \PY{n}{states}\PY{p}{[}\PY{n}{series\PYZus{}id}\PY{p}{]}\PY{p}{;}
                 
                 \PY{n}{means} \PY{o}{=} \PY{n}{df}\PY{o}{.}\PY{n}{groupby}\PY{p}{(}\PY{p}{[}\PY{l+s+s1}{\PYZsq{}}\PY{l+s+s1}{Regime}\PY{l+s+s1}{\PYZsq{}}\PY{p}{]}\PY{p}{)}\PY{p}{[}\PY{l+s+s1}{\PYZsq{}}\PY{l+s+s1}{Diff}\PY{l+s+s1}{\PYZsq{}}\PY{p}{]}\PY{o}{.}\PY{n}{mean}\PY{p}{(}\PY{p}{)}\PY{p}{;} \PY{c+c1}{\PYZsh{} Get means of all assigned states}
                 
                 \PY{n}{lst\PYZus{}1} \PY{o}{=} \PY{n}{means}\PY{o}{.}\PY{n}{index}\PY{o}{.}\PY{n}{tolist}\PY{p}{(}\PY{p}{)}\PY{p}{;}
                 \PY{n}{lst\PYZus{}2} \PY{o}{=} \PY{n}{means}\PY{o}{.}\PY{n}{sort\PYZus{}values}\PY{p}{(}\PY{p}{)}\PY{o}{.}\PY{n}{index}\PY{o}{.}\PY{n}{tolist}\PY{p}{(}\PY{p}{)}\PY{p}{;}
         
                 \PY{n}{map\PYZus{}regimes} \PY{o}{=} \PY{n+nb}{dict}\PY{p}{(}\PY{n+nb}{zip}\PY{p}{(}\PY{n}{lst\PYZus{}2}\PY{p}{,} \PY{n}{lst\PYZus{}1}\PY{p}{)}\PY{p}{)}\PY{p}{;}
                 \PY{n}{df}\PY{p}{[}\PY{l+s+s1}{\PYZsq{}}\PY{l+s+s1}{Regime}\PY{l+s+s1}{\PYZsq{}}\PY{p}{]} \PY{o}{=} \PY{n}{df}\PY{p}{[}\PY{l+s+s1}{\PYZsq{}}\PY{l+s+s1}{Regime}\PY{l+s+s1}{\PYZsq{}}\PY{p}{]}\PY{o}{.}\PY{n}{map}\PY{p}{(}\PY{n}{map\PYZus{}regimes}\PY{p}{)}\PY{p}{;}
                 
                 
                 \PY{n}{cmap}   \PY{o}{=} \PY{n}{ListedColormap}\PY{p}{(}\PY{p}{[}\PY{l+s+s1}{\PYZsq{}}\PY{l+s+s1}{r}\PY{l+s+s1}{\PYZsq{}}\PY{p}{,}\PY{l+s+s1}{\PYZsq{}}\PY{l+s+s1}{b}\PY{l+s+s1}{\PYZsq{}}\PY{p}{,}\PY{l+s+s1}{\PYZsq{}}\PY{l+s+s1}{g}\PY{l+s+s1}{\PYZsq{}}\PY{p}{]}\PY{p}{,}\PY{l+s+s1}{\PYZsq{}}\PY{l+s+s1}{indexed}\PY{l+s+s1}{\PYZsq{}}\PY{p}{)}\PY{p}{;}
                 \PY{n}{norm}   \PY{o}{=} \PY{n}{BoundaryNorm}\PY{p}{(}\PY{n+nb}{range}\PY{p}{(}\PY{l+m+mi}{3} \PY{o}{+} \PY{l+m+mi}{1}\PY{p}{)}\PY{p}{,} \PY{n}{cmap}\PY{o}{.}\PY{n}{N}\PY{p}{)}\PY{p}{;}
                 \PY{n}{inxval} \PY{o}{=} \PY{n}{mdates}\PY{o}{.}\PY{n}{date2num}\PY{p}{(}\PY{n}{df}\PY{p}{[}\PY{n}{series\PYZus{}id}\PY{p}{]}\PY{o}{.}\PY{n}{index}\PY{o}{.}\PY{n}{to\PYZus{}pydatetime}\PY{p}{(}\PY{p}{)}\PY{p}{)}\PY{p}{;}
                 \PY{n}{points} \PY{o}{=} \PY{n}{np}\PY{o}{.}\PY{n}{array}\PY{p}{(}\PY{p}{[}\PY{n}{inxval}\PY{p}{,} \PY{n}{df}\PY{p}{[}\PY{n}{series\PYZus{}id}\PY{p}{]}\PY{p}{]}\PY{p}{)}\PY{o}{.}\PY{n}{T}\PY{o}{.}\PY{n}{reshape}\PY{p}{(}\PY{o}{\PYZhy{}}\PY{l+m+mi}{1}\PY{p}{,} \PY{l+m+mi}{1}\PY{p}{,} \PY{l+m+mi}{2}\PY{p}{)}\PY{p}{;}
                 \PY{n}{segments} \PY{o}{=} \PY{n}{np}\PY{o}{.}\PY{n}{concatenate}\PY{p}{(}\PY{p}{[}\PY{n}{points}\PY{p}{[}\PY{p}{:}\PY{o}{\PYZhy{}}\PY{l+m+mi}{1}\PY{p}{]}\PY{p}{,} \PY{n}{points}\PY{p}{[}\PY{l+m+mi}{1}\PY{p}{:}\PY{p}{]}\PY{p}{]}\PY{p}{,} \PY{n}{axis}\PY{o}{=}\PY{l+m+mi}{1}\PY{p}{)}\PY{p}{;}
         
                 \PY{n}{lc} \PY{o}{=} \PY{n}{LineCollection}\PY{p}{(}\PY{n}{segments}\PY{p}{,} \PY{n}{cmap}\PY{o}{=}\PY{n}{cmap}\PY{p}{,} \PY{n}{norm}\PY{o}{=}\PY{n}{norm}\PY{p}{)}\PY{p}{;}
                 \PY{n}{lc}\PY{o}{.}\PY{n}{set\PYZus{}array}\PY{p}{(}\PY{n}{df}\PY{p}{[}\PY{l+s+s1}{\PYZsq{}}\PY{l+s+s1}{Regime}\PY{l+s+s1}{\PYZsq{}}\PY{p}{]}\PY{p}{)}\PY{p}{;}
                 \PY{n}{plt}\PY{o}{.}\PY{n}{gca}\PY{p}{(}\PY{p}{)}\PY{o}{.}\PY{n}{add\PYZus{}collection}\PY{p}{(}\PY{n}{lc}\PY{p}{)}\PY{p}{;}
                 \PY{n}{plt}\PY{o}{.}\PY{n}{xlim}\PY{p}{(}\PY{n}{df}\PY{p}{[}\PY{n}{series\PYZus{}id}\PY{p}{]}\PY{o}{.}\PY{n}{index}\PY{o}{.}\PY{n}{min}\PY{p}{(}\PY{p}{)}\PY{p}{,} \PY{n}{df}\PY{p}{[}\PY{n}{series\PYZus{}id}\PY{p}{]}\PY{o}{.}\PY{n}{index}\PY{o}{.}\PY{n}{max}\PY{p}{(}\PY{p}{)}\PY{p}{)}\PY{p}{;}
                 \PY{n}{plt}\PY{o}{.}\PY{n}{ylim}\PY{p}{(}\PY{n}{df}\PY{p}{[}\PY{n}{series\PYZus{}id}\PY{p}{]}\PY{o}{.}\PY{n}{min}\PY{p}{(}\PY{p}{)}\PY{p}{,} \PY{n}{df}\PY{p}{[}\PY{n}{series\PYZus{}id}\PY{p}{]}\PY{o}{.}\PY{n}{max}\PY{p}{(}\PY{p}{)}\PY{p}{)}\PY{p}{;}
         
                 \PY{n}{r\PYZus{}patch} \PY{o}{=} \PY{n}{mpatches}\PY{o}{.}\PY{n}{Patch}\PY{p}{(}\PY{n}{color}\PY{o}{=}\PY{l+s+s1}{\PYZsq{}}\PY{l+s+s1}{red}\PY{l+s+s1}{\PYZsq{}}\PY{p}{,} \PY{n}{label}\PY{o}{=}\PY{l+s+s1}{\PYZsq{}}\PY{l+s+s1}{Bear}\PY{l+s+s1}{\PYZsq{}}\PY{p}{)}\PY{p}{;}
                 \PY{n}{g\PYZus{}patch} \PY{o}{=} \PY{n}{mpatches}\PY{o}{.}\PY{n}{Patch}\PY{p}{(}\PY{n}{color}\PY{o}{=}\PY{l+s+s1}{\PYZsq{}}\PY{l+s+s1}{green}\PY{l+s+s1}{\PYZsq{}}\PY{p}{,} \PY{n}{label}\PY{o}{=}\PY{l+s+s1}{\PYZsq{}}\PY{l+s+s1}{Bull}\PY{l+s+s1}{\PYZsq{}}\PY{p}{)}\PY{p}{;}
                 \PY{n}{b\PYZus{}patch} \PY{o}{=} \PY{n}{mpatches}\PY{o}{.}\PY{n}{Patch}\PY{p}{(}\PY{n}{color}\PY{o}{=}\PY{l+s+s1}{\PYZsq{}}\PY{l+s+s1}{blue}\PY{l+s+s1}{\PYZsq{}}\PY{p}{,} \PY{n}{label}\PY{o}{=}\PY{l+s+s1}{\PYZsq{}}\PY{l+s+s1}{Stagnant}\PY{l+s+s1}{\PYZsq{}}\PY{p}{)}\PY{p}{;}
         
                 \PY{n}{plt}\PY{o}{.}\PY{n}{legend}\PY{p}{(}\PY{n}{handles}\PY{o}{=}\PY{p}{[}\PY{n}{r\PYZus{}patch}\PY{p}{,} \PY{n}{g\PYZus{}patch}\PY{p}{,} \PY{n}{b\PYZus{}patch}\PY{p}{]}\PY{p}{)}\PY{p}{;}
         
                 \PY{n}{name} \PY{o}{=} \PY{l+s+s2}{\PYZdq{}}\PY{l+s+s2}{./plots/}\PY{l+s+s2}{\PYZdq{}} \PY{o}{+} \PY{n}{series\PYZus{}id}\PY{o}{.}\PY{n}{replace}\PY{p}{(}\PY{l+s+s2}{\PYZdq{}}\PY{l+s+s2}{.}\PY{l+s+s2}{\PYZdq{}}\PY{p}{,} \PY{l+s+s2}{\PYZdq{}}\PY{l+s+s2}{\PYZus{}}\PY{l+s+s2}{\PYZdq{}}\PY{p}{)} \PY{o}{+} \PY{l+s+s2}{\PYZdq{}}\PY{l+s+s2}{\PYZus{}VALIDATION.png}\PY{l+s+s2}{\PYZdq{}}\PY{p}{;}
         
                 \PY{n}{plt}\PY{o}{.}\PY{n}{savefig}\PY{p}{(}\PY{n}{name}\PY{p}{)}\PY{p}{;}
                 \PY{n}{plt}\PY{o}{.}\PY{n}{close}\PY{p}{(}\PY{p}{)}\PY{p}{;}
\end{Verbatim}


    Now, we would be predicting the validation sets (forecast for Spot Crude
Oil price, WTI Monthly).

For that, we have to drop the forecast column and then do an inference
on the model.

    \begin{Verbatim}[commandchars=\\\{\}]
{\color{incolor}In [{\color{incolor}33}]:} \PY{n}{vald\PYZus{}real} \PY{o}{=} \PY{n}{states}\PY{p}{[}\PY{l+s+s1}{\PYZsq{}}\PY{l+s+s1}{WTISPLC}\PY{l+s+s1}{\PYZsq{}}\PY{p}{]}\PY{o}{.}\PY{n}{as\PYZus{}matrix}\PY{p}{(}\PY{p}{)}\PY{p}{;} \PY{c+c1}{\PYZsh{} Record real data observation, to be compared with the predicted one}
         \PY{n}{vald\PYZus{}data\PYZus{}new} \PY{o}{=} \PY{n}{states}\PY{o}{.}\PY{n}{drop}\PY{p}{(}\PY{l+s+s1}{\PYZsq{}}\PY{l+s+s1}{forecast}\PY{l+s+s1}{\PYZsq{}}\PY{p}{,} \PY{n}{axis}\PY{o}{=}\PY{l+m+mi}{1}\PY{p}{)}\PY{p}{;} \PY{c+c1}{\PYZsh{} Drop the real data observation so that it does not bias prediction}
         
         \PY{n}{vald\PYZus{}prediction} \PY{o}{=} \PY{n}{model}\PY{o}{.}\PY{n}{predict}\PY{p}{(}\PY{n}{vald\PYZus{}data\PYZus{}new}\PY{p}{)}\PY{p}{;} \PY{c+c1}{\PYZsh{} Inference on the constructed graphical model}
         \PY{n}{pred\PYZus{}value\PYZus{}vald} \PY{o}{=} \PY{n}{vald\PYZus{}prediction}\PY{p}{[}\PY{l+s+s1}{\PYZsq{}}\PY{l+s+s1}{forecast}\PY{l+s+s1}{\PYZsq{}}\PY{p}{]}\PY{o}{.}\PY{n}{as\PYZus{}matrix}\PY{p}{(}\PY{p}{)}\PY{p}{;} \PY{c+c1}{\PYZsh{} Retrieve it as an array so we can compare with real value}
\end{Verbatim}


    \begin{Verbatim}[commandchars=\\\{\}]
{\color{incolor}In [{\color{incolor}34}]:} \PY{n+nb}{print}\PY{p}{(}\PY{l+s+s2}{\PYZdq{}}\PY{l+s+se}{\PYZbs{}n}\PY{l+s+s2}{Predicted Value: }\PY{l+s+s2}{\PYZdq{}}\PY{p}{)}\PY{p}{;}
         \PY{n+nb}{print}\PY{p}{(}\PY{n}{pred\PYZus{}value\PYZus{}vald}\PY{p}{)}\PY{p}{;}
         \PY{n+nb}{print}\PY{p}{(}\PY{l+s+s2}{\PYZdq{}}\PY{l+s+se}{\PYZbs{}n}\PY{l+s+s2}{Real Value: }\PY{l+s+s2}{\PYZdq{}}\PY{p}{)}\PY{p}{;}
         \PY{n+nb}{print}\PY{p}{(}\PY{n}{vald\PYZus{}real}\PY{p}{)}\PY{p}{;}
         
         \PY{n}{error} \PY{o}{=} \PY{n}{np}\PY{o}{.}\PY{n}{mean}\PY{p}{(}\PY{n}{vald\PYZus{}real} \PY{o}{!=}  \PY{n}{np}\PY{o}{.}\PY{n}{roll}\PY{p}{(}\PY{n}{pred\PYZus{}value\PYZus{}vald}\PY{p}{,} \PY{l+m+mi}{1}\PY{p}{)}\PY{p}{)}\PY{p}{;}
         \PY{c+c1}{\PYZsh{}error = np.mean(vald\PYZus{}real != pred\PYZus{}value\PYZus{}vald);}
         \PY{n+nb}{print}\PY{p}{(}\PY{l+s+s2}{\PYZdq{}}\PY{l+s+se}{\PYZbs{}n}\PY{l+s+s2}{Error: }\PY{l+s+s2}{\PYZdq{}}\PY{p}{)}\PY{p}{;}
         \PY{n+nb}{print}\PY{p}{(}\PY{n}{error} \PY{o}{*} \PY{l+m+mi}{100}\PY{p}{)}\PY{p}{;}
\end{Verbatim}


    \begin{Verbatim}[commandchars=\\\{\}]

Predicted Value: 
[0 1 1 2 2 2 2 2 2 2 1 1 1 1 2 1 1 1 1 1 1 1 2 2 2 2 1 1]

Real Value: 
[1 1 1 1 2 2 2 2 2 2 0 1 1 1 2 0 0 0 0 0 0 1 2 2 2 2 0 0]

Error: 
42.857142857142854

    \end{Verbatim}

    Now, we are going to trade on the predictions and compare the
performance of the algorithm on returns, beginning with one oil share.

    If we have achieved reasonable satisfaction of results on the
\textbf{validation set}, we can now move to using the \textbf{test} set,
and check the performance of our algorithm on that.

The \textbf{validation step} is to adjust the model if the error is too
high. In this case, we can start again by learning the Bayesian model
via the Hill Climbing method and observe the change in performance. We
should however, \textbf{not} use the \textbf{test set} unless and until
we are satisfied with the performance of the model on the
\textbf{validation set}.

One way of assessing the quality of a network structure is by examining
the \textbf{connectedness} of the graph, ensuring there are almost no
forests, and \textbf{the variables which are being forecasted} are part
of a denser tree. If we happen to see disconnected trees, we should run
our Hill Climbing method again to ensure that the Hill Climber
\textbf{converges} to either ideally a \textbf{global maximum} or
atleast a better \textbf{local maximum}.

\hypertarget{testing}{%
\subsubsection{Testing}\label{testing}}

Similar to what we did with the validation data set, we first have to
discretise the test dataset, and predict it on the \emph{BayesianModel}.

    \begin{Verbatim}[commandchars=\\\{\}]
{\color{incolor}In [{\color{incolor}35}]:} \PY{n}{discrete\PYZus{}test} \PY{o}{=} \PY{n}{pd}\PY{o}{.}\PY{n}{DataFrame}\PY{p}{(}\PY{n}{index} \PY{o}{=} \PY{n}{test\PYZus{}data}\PY{p}{[}\PY{l+m+mi}{1}\PY{p}{:}\PY{p}{]}\PY{o}{.}\PY{n}{index}\PY{p}{)}\PY{p}{;}
         
         \PY{k}{for} \PY{n}{series\PYZus{}id} \PY{o+ow}{in} \PY{n}{datasets}\PY{p}{:}
             \PY{n}{path} \PY{o}{=} \PY{l+s+s2}{\PYZdq{}}\PY{l+s+s2}{./hmms/}\PY{l+s+s2}{\PYZdq{}} \PY{o}{+} \PY{n}{series\PYZus{}id}\PY{o}{.}\PY{n}{replace}\PY{p}{(}\PY{l+s+s2}{\PYZdq{}}\PY{l+s+s2}{.}\PY{l+s+s2}{\PYZdq{}}\PY{p}{,} \PY{l+s+s2}{\PYZdq{}}\PY{l+s+s2}{\PYZus{}}\PY{l+s+s2}{\PYZdq{}}\PY{p}{)} \PY{o}{+} \PY{l+s+s2}{\PYZdq{}}\PY{l+s+s2}{.npz}\PY{l+s+s2}{\PYZdq{}}\PY{p}{;}
             \PY{k}{if} \PY{n}{series\PYZus{}id} \PY{o}{==} \PY{l+s+s1}{\PYZsq{}}\PY{l+s+s1}{forecast}\PY{l+s+s1}{\PYZsq{}}\PY{p}{:}
                 \PY{n}{dhmm} \PY{o}{=} \PY{n}{hmms}\PY{o}{.}\PY{n}{DtHMM}\PY{o}{.}\PY{n}{from\PYZus{}file}\PY{p}{(}\PY{l+s+s1}{\PYZsq{}}\PY{l+s+s1}{./hmms/WTISPLC.npz}\PY{l+s+s1}{\PYZsq{}}\PY{p}{)}\PY{p}{;}
             \PY{k}{else}\PY{p}{:}
                 \PY{n}{dhmm} \PY{o}{=} \PY{n}{hmms}\PY{o}{.}\PY{n}{DtHMM}\PY{o}{.}\PY{n}{from\PYZus{}file}\PY{p}{(}\PY{n}{path}\PY{p}{)}\PY{p}{;}
             \PY{n}{data\PYZus{}diff} \PY{o}{=}  \PY{n}{test\PYZus{}data}\PY{p}{[}\PY{n}{series\PYZus{}id}\PY{p}{]}\PY{o}{.}\PY{n}{diff}\PY{p}{(}\PY{p}{)}\PY{p}{[}\PY{l+m+mi}{1}\PY{p}{:}\PY{p}{]}\PY{p}{;}
             \PY{n}{emit\PYZus{}seq} \PY{o}{=} \PY{n}{np}\PY{o}{.}\PY{n}{array}\PY{p}{(}\PY{n}{data\PYZus{}diff}\PY{o}{.}\PY{n}{apply}\PY{p}{(}\PY{k}{lambda} \PY{n}{x}\PY{p}{:} \PY{l+m+mi}{1} \PY{k}{if} \PY{n}{x} \PY{o}{\PYZgt{}} \PY{l+m+mi}{0} \PY{k}{else} \PY{l+m+mi}{0}\PY{p}{)}\PY{o}{.}\PY{n}{values}\PY{p}{)}\PY{p}{;}
             \PY{p}{(} \PY{n}{log\PYZus{}prob}\PY{p}{,} \PY{n}{s\PYZus{}seq} \PY{p}{)} \PY{o}{=}  \PY{n}{dhmm}\PY{o}{.}\PY{n}{viterbi}\PY{p}{(}\PY{n}{emit\PYZus{}seq}\PY{p}{)}\PY{p}{;}
             \PY{n}{discrete\PYZus{}test}\PY{p}{[}\PY{n}{series\PYZus{}id}\PY{p}{]} \PY{o}{=} \PY{n}{s\PYZus{}seq}\PY{p}{;}
         
         \PY{n}{discrete\PYZus{}test}\PY{o}{.}\PY{n}{to\PYZus{}csv}\PY{p}{(}\PY{l+s+s2}{\PYZdq{}}\PY{l+s+s2}{./data/test\PYZus{}data.csv}\PY{l+s+s2}{\PYZdq{}}\PY{p}{)}\PY{p}{;} \PY{c+c1}{\PYZsh{} Saving to CSV}
\end{Verbatim}


    We would now import the (discretised) test dataframe, remove the column
containing forecast column as we are predicting it, and input it in our
learnt model. We would compare the output to the real values and make a
final conclusion of the reliability of the model.

    \begin{Verbatim}[commandchars=\\\{\}]
{\color{incolor}In [{\color{incolor}36}]:} \PY{n}{discrete\PYZus{}test} \PY{o}{=} \PY{n}{pd}\PY{o}{.}\PY{n}{read\PYZus{}csv}\PY{p}{(}\PY{l+s+s2}{\PYZdq{}}\PY{l+s+s2}{./data/test\PYZus{}data.csv}\PY{l+s+s2}{\PYZdq{}}\PY{p}{,} \PY{n}{index\PYZus{}col}\PY{o}{=}\PY{l+m+mi}{0}\PY{p}{)}\PY{p}{;}
         
         \PY{n}{test\PYZus{}real} \PY{o}{=} \PY{n}{discrete\PYZus{}test}\PY{p}{[}\PY{l+s+s1}{\PYZsq{}}\PY{l+s+s1}{WTISPLC}\PY{l+s+s1}{\PYZsq{}}\PY{p}{]}\PY{o}{.}\PY{n}{as\PYZus{}matrix}\PY{p}{(}\PY{p}{)}\PY{p}{;} \PY{c+c1}{\PYZsh{} Record real data observation, to be compared with the predicted one}
         \PY{n}{test\PYZus{}data\PYZus{}new} \PY{o}{=} \PY{n}{discrete\PYZus{}test}\PY{o}{.}\PY{n}{drop}\PY{p}{(}\PY{l+s+s1}{\PYZsq{}}\PY{l+s+s1}{forecast}\PY{l+s+s1}{\PYZsq{}}\PY{p}{,} \PY{n}{axis}\PY{o}{=}\PY{l+m+mi}{1}\PY{p}{)}\PY{p}{;} \PY{c+c1}{\PYZsh{} Drop the real data observation so that it does not bias prediction}
         
         \PY{n}{test\PYZus{}prediction} \PY{o}{=} \PY{n}{model}\PY{o}{.}\PY{n}{predict}\PY{p}{(}\PY{n}{test\PYZus{}data\PYZus{}new}\PY{p}{)}\PY{p}{;} \PY{c+c1}{\PYZsh{} Inference on the constructed graphical model}
         \PY{n}{pred\PYZus{}value\PYZus{}test} \PY{o}{=} \PY{n}{test\PYZus{}prediction}\PY{p}{[}\PY{l+s+s1}{\PYZsq{}}\PY{l+s+s1}{forecast}\PY{l+s+s1}{\PYZsq{}}\PY{p}{]}\PY{o}{.}\PY{n}{as\PYZus{}matrix}\PY{p}{(}\PY{p}{)}\PY{p}{;} \PY{c+c1}{\PYZsh{} Retrieve it as an array so we can compare with real value}
\end{Verbatim}


    \begin{Verbatim}[commandchars=\\\{\}]
{\color{incolor}In [{\color{incolor}37}]:} \PY{n+nb}{print}\PY{p}{(}\PY{l+s+s2}{\PYZdq{}}\PY{l+s+se}{\PYZbs{}n}\PY{l+s+s2}{Predicted Value: }\PY{l+s+s2}{\PYZdq{}}\PY{p}{)}\PY{p}{;}
         \PY{n+nb}{print}\PY{p}{(}\PY{n}{pred\PYZus{}value\PYZus{}test}\PY{p}{)}\PY{p}{;} \PY{c+c1}{\PYZsh{} This is the price, not the forecast}
         \PY{n+nb}{print}\PY{p}{(}\PY{l+s+s2}{\PYZdq{}}\PY{l+s+se}{\PYZbs{}n}\PY{l+s+s2}{Real Value: }\PY{l+s+s2}{\PYZdq{}}\PY{p}{)}\PY{p}{;}
         \PY{n+nb}{print}\PY{p}{(}\PY{n}{test\PYZus{}real}\PY{p}{)}\PY{p}{;}
         
         \PY{n}{error} \PY{o}{=} \PY{n}{np}\PY{o}{.}\PY{n}{mean}\PY{p}{(}\PY{n}{test\PYZus{}real} \PY{o}{!=} \PY{n}{np}\PY{o}{.}\PY{n}{roll}\PY{p}{(}\PY{n}{pred\PYZus{}value\PYZus{}test}\PY{p}{,} \PY{l+m+mi}{1}\PY{p}{)}\PY{p}{)}\PY{p}{;} \PY{c+c1}{\PYZsh{} Shift to get forecast}
         \PY{c+c1}{\PYZsh{}error = np.mean(test\PYZus{}real != pred\PYZus{}value\PYZus{}test); \PYZsh{} Shift to get forecast}
         \PY{n+nb}{print}\PY{p}{(}\PY{l+s+s2}{\PYZdq{}}\PY{l+s+se}{\PYZbs{}n}\PY{l+s+s2}{Error: }\PY{l+s+s2}{\PYZdq{}}\PY{p}{)}\PY{p}{;}
         \PY{n+nb}{print}\PY{p}{(}\PY{n}{error} \PY{o}{*} \PY{l+m+mi}{100}\PY{p}{)}\PY{p}{;}
\end{Verbatim}


    \begin{Verbatim}[commandchars=\\\{\}]

Predicted Value: 
[0 2 0 0 1 1 1 1 2 2 2 2 2 1 1 2 1 2 2 1 1 1 1 2 1 1 1 1]

Real Value: 
[1 2 0 0 0 1 1 1 1 2 2 2 2 0 1 1 1 2 2 0 0 1 1 1 1 1 1 1]

Error: 
28.57142857142857

    \end{Verbatim}

    Now, we can use the discretised price predictions to construct a simple
trading algorithm, which takes `0' as a shorting signal, `2' for a long
position, and `1' for no action. We assume we start with one barrel of
oil and we would be trading it to acquire a larger unit.

    \begin{Verbatim}[commandchars=\\\{\}]
{\color{incolor}In [{\color{incolor}51}]:} \PY{n}{test\PYZus{}price} \PY{o}{=} \PY{n}{pd}\PY{o}{.}\PY{n}{DataFrame}\PY{p}{(}\PY{n}{test\PYZus{}data}\PY{p}{[}\PY{l+s+s1}{\PYZsq{}}\PY{l+s+s1}{WTISPLC}\PY{l+s+s1}{\PYZsq{}}\PY{p}{]}\PY{p}{,} \PY{n}{columns}\PY{o}{=}\PY{p}{[}\PY{l+s+s1}{\PYZsq{}}\PY{l+s+s1}{WTISPLC}\PY{l+s+s1}{\PYZsq{}}\PY{p}{]}\PY{p}{)}\PY{p}{;}
         \PY{n}{test\PYZus{}signal} \PY{o}{=} \PY{n}{pd}\PY{o}{.}\PY{n}{DataFrame}\PY{p}{(}\PY{n}{test\PYZus{}prediction}\PY{p}{,} \PY{n}{columns}\PY{o}{=}\PY{p}{[}\PY{l+s+s1}{\PYZsq{}}\PY{l+s+s1}{forecast}\PY{l+s+s1}{\PYZsq{}}\PY{p}{]}\PY{p}{)}\PY{p}{;}
         \PY{n}{test\PYZus{}sheet} \PY{o}{=} \PY{n}{pd}\PY{o}{.}\PY{n}{concat}\PY{p}{(}\PY{p}{[}\PY{n}{test\PYZus{}price}\PY{p}{,} \PY{n}{test\PYZus{}signal}\PY{p}{]}\PY{p}{,} \PY{n}{axis}\PY{o}{=}\PY{l+m+mi}{1}\PY{p}{,} \PY{n}{join}\PY{o}{=}\PY{l+s+s1}{\PYZsq{}}\PY{l+s+s1}{inner}\PY{l+s+s1}{\PYZsq{}}\PY{p}{)}\PY{p}{;}
         
         \PY{n}{trades} \PY{o}{=} \PY{p}{[}\PY{n}{test\PYZus{}sheet}\PY{p}{[}\PY{l+s+s1}{\PYZsq{}}\PY{l+s+s1}{WTISPLC}\PY{l+s+s1}{\PYZsq{}}\PY{p}{]}\PY{o}{.}\PY{n}{iloc}\PY{p}{[}\PY{l+m+mi}{0}\PY{p}{]}\PY{p}{]}\PY{p}{;}
         
         \PY{n}{position} \PY{o}{=} \PY{k+kc}{False}\PY{p}{;} \PY{c+c1}{\PYZsh{} True for Long, False for Short}
         
         \PY{k}{for} \PY{n}{i} \PY{o+ow}{in} \PY{n+nb}{range}\PY{p}{(}\PY{n+nb}{len}\PY{p}{(}\PY{n}{test\PYZus{}sheet}\PY{p}{)}\PY{o}{\PYZhy{}}\PY{l+m+mi}{1}\PY{p}{)}\PY{p}{:}
             \PY{k}{if} \PY{n}{test\PYZus{}sheet}\PY{p}{[}\PY{l+s+s1}{\PYZsq{}}\PY{l+s+s1}{forecast}\PY{l+s+s1}{\PYZsq{}}\PY{p}{]}\PY{o}{.}\PY{n}{iloc}\PY{p}{[}\PY{n}{i}\PY{o}{+}\PY{l+m+mi}{1}\PY{p}{]} \PY{o}{==} \PY{l+m+mi}{0}\PY{p}{:}
                 \PY{n}{trades}\PY{o}{.}\PY{n}{append}\PY{p}{(}\PY{n}{trades}\PY{p}{[}\PY{o}{\PYZhy{}}\PY{l+m+mi}{1}\PY{p}{]}\PY{p}{)}\PY{p}{;}
             \PY{k}{elif} \PY{n}{test\PYZus{}sheet}\PY{p}{[}\PY{l+s+s1}{\PYZsq{}}\PY{l+s+s1}{forecast}\PY{l+s+s1}{\PYZsq{}}\PY{p}{]}\PY{o}{.}\PY{n}{iloc}\PY{p}{[}\PY{n}{i}\PY{o}{+}\PY{l+m+mi}{1}\PY{p}{]} \PY{o}{==} \PY{l+m+mi}{2}\PY{p}{:}
                 \PY{k}{if} \PY{n}{position} \PY{o}{==} \PY{k+kc}{False}\PY{p}{:}
                     \PY{n}{position} \PY{o}{=} \PY{k+kc}{True}\PY{p}{;}
                     \PY{n}{trades}\PY{o}{.}\PY{n}{append}\PY{p}{(}\PY{n}{trades}\PY{p}{[}\PY{o}{\PYZhy{}}\PY{l+m+mi}{1}\PY{p}{]}\PY{p}{)}\PY{p}{;}
                 \PY{k}{else}\PY{p}{:}
                     \PY{n}{trades}\PY{o}{.}\PY{n}{append}\PY{p}{(}\PY{n}{test\PYZus{}sheet}\PY{p}{[}\PY{l+s+s1}{\PYZsq{}}\PY{l+s+s1}{WTISPLC}\PY{l+s+s1}{\PYZsq{}}\PY{p}{]}\PY{o}{.}\PY{n}{iloc}\PY{p}{[}\PY{n}{i}\PY{o}{+}\PY{l+m+mi}{1}\PY{p}{]}\PY{p}{)}\PY{p}{;}
             \PY{k}{else}\PY{p}{:}
                 \PY{n}{trades}\PY{o}{.}\PY{n}{append}\PY{p}{(}\PY{n}{trades}\PY{p}{[}\PY{o}{\PYZhy{}}\PY{l+m+mi}{1}\PY{p}{]}\PY{p}{)}\PY{p}{;}
         
         \PY{n}{test\PYZus{}performance} \PY{o}{=} \PY{n}{pd}\PY{o}{.}\PY{n}{DataFrame}\PY{p}{(}\PY{n}{trades}\PY{p}{,} \PY{n}{index} \PY{o}{=} \PY{n}{test\PYZus{}signal}\PY{o}{.}\PY{n}{index}\PY{p}{,} \PY{n}{columns}\PY{o}{=}\PY{p}{[}\PY{l+s+s1}{\PYZsq{}}\PY{l+s+s1}{performance}\PY{l+s+s1}{\PYZsq{}}\PY{p}{]}\PY{p}{)}\PY{p}{;}
         \PY{n}{test\PYZus{}sheet} \PY{o}{=} \PY{n}{pd}\PY{o}{.}\PY{n}{concat}\PY{p}{(}\PY{p}{[}\PY{n}{test\PYZus{}sheet}\PY{p}{,} \PY{n}{test\PYZus{}performance}\PY{p}{]}\PY{p}{,} \PY{n}{axis}\PY{o}{=}\PY{l+m+mi}{1}\PY{p}{,} \PY{n}{join}\PY{o}{=}\PY{l+s+s1}{\PYZsq{}}\PY{l+s+s1}{inner}\PY{l+s+s1}{\PYZsq{}}\PY{p}{)}\PY{p}{;}
         
         \PY{n}{plt}\PY{o}{.}\PY{n}{plot}\PY{p}{(}\PY{n}{test\PYZus{}sheet}\PY{p}{[}\PY{l+s+s1}{\PYZsq{}}\PY{l+s+s1}{WTISPLC}\PY{l+s+s1}{\PYZsq{}}\PY{p}{]}\PY{p}{,} \PY{l+s+s1}{\PYZsq{}}\PY{l+s+s1}{r}\PY{l+s+s1}{\PYZsq{}}\PY{p}{)}\PY{p}{;}
         \PY{n}{plt}\PY{o}{.}\PY{n}{plot}\PY{p}{(}\PY{n}{test\PYZus{}sheet}\PY{p}{[}\PY{l+s+s1}{\PYZsq{}}\PY{l+s+s1}{performance}\PY{l+s+s1}{\PYZsq{}}\PY{p}{]}\PY{p}{,} \PY{l+s+s1}{\PYZsq{}}\PY{l+s+s1}{g}\PY{l+s+s1}{\PYZsq{}}\PY{p}{)}\PY{p}{;}
\end{Verbatim}


    \begin{center}
    \adjustimage{max size={0.9\linewidth}{0.9\paperheight}}{Implementation_files/Implementation_78_0.png}
    \end{center}
    { \hspace*{\fill} \\}
    As we can see, the algorithm has successfully managed to hedge the risk in the early quarter of 2015, but did end up shorting in a bull run in late 2017/early 2018. 

    % Add a bibliography block to the postdoc
    
    
    
    \end{document}
