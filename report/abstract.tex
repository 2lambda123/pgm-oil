% (This file is included by thesis.tex; you do not latex it by itself.)

\begin{abstract}



% The text of the abstract goes here.  If you need to use a \section
% command you will need to use \section*, \subsection*, etc. so that
% you don't get any numbering.  You probably won't be using any of
% these commands in the abstract anyway.

The dissertation investigates the application of Probabilistic Graphical Models (PGMs) in forecasting the price of Crude Oil. \\

This research is important because crude oil plays a very pivotal role in the global economy hence is a very critical macroeconomic indicator of the industrial growth. Given the vast amount of macroeconomic factors affecting the price of crude oil such as supply of oil from OPEC countries, demand of oil from OECD countries, geopolitical and geoeconomic changes among many other variables - probabilistic graphical models (PGMs) allow us to understand by learning the graphical structure. This dissertation proposes condensing data numerous Crude Oil factors into a graphical model in the attempt of creating a accurate forecast of the price of crude oil. \\

The research project experiments with using different libraries in Python in order to construct models of the crude oil market. The experiments in this thesis investigate three main challenges commonly presented while trading oil in the financial markets. The first challenge it investigates is the process of learning the structure of the oil markets; thus allowing crude oil traders to understand the different physical market factors and macroeconomic indicators affecting crude oil markets and how they are \textit{causaly} related. The second challenge it solves is the exploration and exploitation of the available data and the learnt structure in predicting the behaviour of the oil markets. The third challenge it investigates is how to validate the performance and reliability of the constructed model in order for it to be deployed in the financial markets. \\

A design and implementation of a probabilistic framework for forecasting the price of crude oil is also presented as part of the research.\\

\section*{Contributions to Science}

The primary contribution of this dissertation to the Quantitative Investment Strategy research are the robust Bayesian-based models for the oil market to handle challenges in forecasting the spot price of crude oil in real world trading environment for portfolio managers. Computational Finance and Bayesian Methodologies are proposed to:

\begin{enumerate}
	\item{Identify the different macroeconomic, microeconomic and geopolitical variables that affect the price of oil globally.}
	\item{Determine the causal relationship between these different variables  and creating a probabilistic graphical model to represent the oil market dynamics.}
	\item{Evaluating and testing the performance of this model by stress testing: simulating situations of economic distress in order to assess the reliability of the model.}
\end{enumerate}

This thesis contributes to existing literature in a number of ways: 
\bigskip
First, it allows us to forecast the price of crude oil using a global-macro strategy, using macroeconomic data and physical market factors of the oil markets. Secondly, it proposes a method of learning the structure of the oil markets and therefore allows us to crunch a large amount of data. Lastly, it allows us to understand how the oil markets work and therefore assists economists and energy policy makers to determine which factors to control and what policies to draft in an attempt to prevent the economy from going in a recession and to control climate change by reducing dependency on fossil fuels. 


\end{abstract}
